\documentclass[aspectratio=169,10pt]{beamer}

% Import all packages
%%%%%%%%%%%%%%%%%%%%%%%%%%%%%%%
%     import des packages     %
%%%%%%%%%%%%%%%%%%%%%%%%%%%%%%%
\usepackage[export]{adjustbox}
\usepackage{amsmath,amsfonts,amssymb}
\usepackage{anyfontsize}
\usepackage{array}
\usepackage[french]{babel}
\usepackage{bbm}
\usepackage{colortbl}
\usepackage{comment}
\usepackage{cclicenses}
\usepackage{eqnarray}
\usepackage{eso-pic}
\usepackage{enumerate} % Pour personnaliser les énumération
\usepackage{fancybox}
\usepackage{fancyhdr}
\usepackage{float}
\usepackage[T1]{fontenc} 
\usepackage{forest}
\usepackage{gensymb}
\usepackage{geometry}
\usepackage{glossaries}
\usepackage{graphicx}
\usepackage{hyperref}
\usepackage{ifthen}
\usepackage{import}
\usepackage{indentfirst}
\usepackage[utf8]{inputenc}
\usepackage{lastpage}
\usepackage{libertine}
\usepackage{lipsum}
\usepackage{listings}
\usepackage{lmodern}
\usepackage{mathtools}
\usepackage{mdframed}
\usepackage{multicol}
\usepackage{pdfpages}
\usepackage{pifont}
\usepackage{stmaryrd}
\usepackage{subcaption}
\usepackage{subfiles}
\usepackage{tabularx}
% \usepackage{tcolorbox}
\usepackage[most]{tcolorbox}
\usepackage[absolute,overlay]{textpos} % To position the image
\usepackage{textcomp}
\usepackage{ulem}
\usepackage{wrapfig}
\usepackage{xcolor}


\newcommand{\titre}{Conception et optimisation d'algorithmes de coordination multi-robot}
\newcommand{\titrefooter}{Soutenance - Projet de fin d'études}
\newcommand{\soustitre}{Exploration autonome de réseaux de galerie}
\newcommand{\auteur}{Fabien MATHÉ}
\newcommand{\referent}{M. Mehmet ERSOY}
\newcommand{\institut}{Seatech - MOCA 3A}
\newcommand{\presentationdate}{27 février 2025}


% Thème clair et épuré
\usetheme{Berlin} % Thème Berlin
\usecolortheme{seahorse}
\setbeamertemplate{navigation symbols}{} % Suppression des symboles de navigation

% Personnalisation du thème
\include{personalisation}

% Titre de la présentation
\title[Titre court]{\titre}
\subtitle{\soustitre}
\author{\auteur}
\institute{\institut \newline \titrefooter}
\date{\presentationdate}

\usefonttheme[onlymath]{serif}

\begin{document}

% Page de titre (Timeline non affichée ici)

\setbeamertemplate{headline}{}
\setbeamertemplate{footline}{}

\begin{frame}
    \titlepage
\end{frame}

\setbeamertemplate{headline}[miniframes theme]
\customfootline

% Section 1 : Introduction
\section{Introduction}

\begin{frame}{\textbf{Introduction}}
	\begin{minipage}{0.6\linewidth}
		\begin{itemize}
			\item Fascination pour l'inconnu
			\vspace{0.2cm}
			\item Meilleure compréhension de notre planète
			\vspace{0.2cm}
			\item Défis inhérents de l'exploration souterraine
			\begin{itemize}
				\item Accessibilité
				\vspace{0.2cm}
				\item Communication
				\vspace{0.2cm}
				\item Navigation
			\end{itemize}
			\vspace{0.2cm}
			\item Intérêt scientifique
			\begin{itemize}
				\item Découverte de nouvelles formes de vie
				\vspace{0.2cm}
				\item Structures géologiques préservées
			\end{itemize}
		\end{itemize}
	\end{minipage}
	\hfill
	\begin{minipage}{0.35\linewidth}
		\begin{figure}
			\centering
			\includegraphics[width=0.8\textwidth]{IMAGES/Grotte_de_la_flûte_de_pan.jpeg}
			\caption{Grotte de la flûte de pan (Guilin, Chine). Paul Munhoven \href{https://commons.wikimedia.org/w/index.php?curid=27712205}{\copyright}}
			\label{fig:china_cave}
		\end{figure}
	\end{minipage}

\end{frame}

\begin{frame}{\textbf{Introduction}}

	\begin{minipage}{0.6\linewidth}
		\begin{figure}
			\centering
			\includegraphics[width=0.8\textwidth]{IMAGES/le-lac-des-grottes-lombrives.jpg}
			\caption{Lac de la Grotte de Lombrives - Ariège}
			\label{fig:ariege_cave}
		\end{figure}
	\end{minipage}
	\hfill
	\begin{minipage}{0.35\linewidth}
		\begin{figure}
			\centering
			\includegraphics[width=0.9\textwidth]{IMAGES/Elizabeth.png}
			\caption{Elizabeth, le robot serpent. Nico Zevalios and Chaohui Gong}
			\label{fig:robot_exploration}
		\end{figure}
	\end{minipage}
    
	\textbf{Objectif :}

	\vspace{0.8em}
	\begin{itemize}
		\item Développer des algorithmes de navigation pour des robots autonomes en milieu souterrain.
		\vspace{0.2cm}
		\item Assurer une communication autonome entre des robots sans contrôle externe.
	\end{itemize}
\end{frame}

% Table des matières
\begin{frame}{\textbf{Sommaire}}
    \tableofcontents
\end{frame}




%%%%%%%%%%%%%%%%%%%%%%%%%%%%%%%%%%%%%%%%%%%%%%%%%%%%%%%%%%%%%%%%%%%%%%%%%%%%%%%%
% Section 2
%%%%%%%%%%%%%%%%%%%%%%%%%%%%%%%%%%%%%%%%%%%%%%%%%%%%%%%%%%%%%%%%%%%%%%%%%%%%%%%%
\section{Plannification de trajectoire}

\begin{frame}{\textbf{Tentative de d'optimisation théorique}}
	\textbf{Objectif}
	\vspace{0.5em}
	
	Trouver le plus court chemin permettant d'explorer une carte inconnue
	
	\vspace{0.5em}
	\begin{itemize}
		\item Contraintes :
		\begin{itemize}
			\item Évitement des obstacles
		\end{itemize}
		\vspace{0.2cm}
		\item Simplification :
		\begin{itemize}
			\item Aucune contrainte liée au robot
			\vspace{0.2cm}
			\item Espace 2D
		\end{itemize}
	\end{itemize}
	
\end{frame}

\begin{frame}[t]{\textbf{Définition de l'espace étoilé}}
	\begin{minipage}[t]{0.6\linewidth}
		On défini :
		\begin{equation*}
			\displaystyle
			FV(t) = \left\{ \, (1 - l) \, \mathbf{X}(t) + l\, \mathbf{M}(t, \alpha) \,|\, l \in [0, 1], \alpha \in [0, 2 \pi[ \, \right\}
		\end{equation*}
			
		\begin{equation*}
			\mathbf{M}(t, \alpha) = \mathbf{X}(t) + R(t, \alpha) 
			\begin{pmatrix}
				\cos \alpha & 0\\
				0 & \sin \alpha\\
			\end{pmatrix} 
			\mathbf{X}(t)
		\end{equation*}
	
		Avec $R(t, \alpha)$ la distance minimale entre le robot et le plus proche point d'intersection à un mur, borné par $R_{\max}$.
	\end{minipage}
	\hfill
	\begin{minipage}[t]{0.38\linewidth}
		\begin{figure}
			\centering
			\includegraphics[width=\textwidth]{IMAGES/math_nota.png}
			\caption{Schéma des notations}
			\label{fig:math_notation}
		\end{figure}
	\end{minipage}
	
	\begin{itemize}
		\item $\displaystyle KM = \bigcup_{t=0}^{T} FV(t)$ : La partie de la carte connue par le robot
		\item $EM$ : La partie de la carte explorable par le robot
	\end{itemize}

\end{frame}

\begin{frame}{\textbf{Définition de la fonctionnelle et problèmes rencontrés}}
	\textbf{Fonctionnelle :}

	\begin{equation*}
		\displaystyle
		J(\mathbf{X}(0), T) = \int_{0}^{T} \| \mathbf{\dot{X}}(t) \| dt
	\end{equation*}

	avec pour contrainte $\mathbf{X}(0) = \mathbf{X}_{Robot}$, $\mathbf{X}(T) = \mathbf{X}_{WP}$, $KM(T) = EM$ et $\mathbf{\dot{X}}$ bornée.
	
	\begin{itemize}
		\item Comment garantir $\mathbf{X}(T)=\mathbf{X}_{WP}$ en contraignant $\dot{X}(t)$ ?
		\item Comment choisir le temps T ?
		\begin{itemize}
			\item T doit être suffisamment grand pour atteindre la cible.
			\item Si le robot s'arrête en $X_{WP}$ à $t = t_1$, alors :
			$$
			\displaystyle
			\int_{t_1}^{T} \| \mathbf{\dot{X}}(t) \| dt = 0
			$$
		\end{itemize}
		\item Comment prendre en compte l'exploration de la carte ?
	\end{itemize}
\end{frame}

\begin{frame}{\textbf{Proposition d'une nouvelle méthode}}
	\begin{figure}[H]
		\centering
		\begin{subfigure}[b]{0.3\textwidth}
			\centering
			\includegraphics[width=\textwidth]{IMAGES/methode1.png}
			\caption{Tracer une ligne}
			\label{fig:draw_line}
		\end{subfigure}
		\hfill
		\begin{subfigure}[b]{0.3\textwidth}
			\centering
			\includegraphics[width=\textwidth]{IMAGES/methode2.png}
			\caption{Vérifier si le chemin est libre}
			\label{fig:check_line}
		\end{subfigure}
		\hfill
		\begin{subfigure}[b]{0.3\textwidth}
			\centering
			\includegraphics[width=\textwidth]{IMAGES/methode3.png}
			\caption{Tirer des points normaux}
			\label{fig:shoot_line}
		\end{subfigure}
		\vfill
		\begin{subfigure}[b]{0.3\textwidth}
			\centering
			\includegraphics[width=\textwidth]{IMAGES/methode4.png}
			\caption{Valider le point}
			\label{fig:validate_point}
		\end{subfigure}
		\hfill
		\begin{subfigure}[b]{0.3\textwidth}
			\centering
			\includegraphics[width=\textwidth]{IMAGES/methode5.png}
			\caption{Répéter la méthode}
			\label{fig:repeat_method}
		\end{subfigure}
		\hfill
		\begin{subfigure}[b]{0.3\textwidth}
			\centering
			\includegraphics[width=\textwidth]{IMAGES/methode6.png}
			\caption{Connecter les points}
			\label{fig:idk}
		\end{subfigure}
		\caption{Visualisation de la méthode de recherche de chemin dynamique}
		\label{fig:method_visu}
	\end{figure}
\end{frame}

\begin{frame}{\textbf{Proposition d'une nouvelle méthode}}
    \begin{figure}[H]
		\centering
		\includegraphics[width=0.7\textwidth]{IMAGES/shorten_path.png}
		\caption{Processus de simplification du chemin trouvé}
		\label{fig:path_simp}
	\end{figure}
\end{frame}

\begin{frame}{\textbf{Méthode de vérification}}
    Le chemin le plus court est modélisé par la propagation d'une onde à vitesse constante.\cite{tapia_2016}
	\begin{columns}[t]
		% Column 1
		\begin{column}{0.6\textwidth}
			\begin{itemize}
				\item Utilisation d'un automate cellulaire (CA)
				\vspace{0.2cm}
				\item La structure en réseau définit :
				\begin{itemize}
					\item Les cellules activées
					\vspace{0.1cm}
					\item Les obstacles
					\vspace{0.1cm}
					\item Les sources secondaires d'ondes
					\vspace{0.1cm}
					\item Les espaces vides
				\end{itemize}
				\item Un vecteur de distance suit la progression de l'onde.
				\vspace{0.2cm}
				\item Sources secondaires d'ondes ajoutées pour gérer les obstacles, inspirées du principe de Huygens.
			\end{itemize}
		\end{column}
		% Column 2    
		\begin{column}{0.4\textwidth}
			\begin{figure}
				\centering
				\includegraphics[width=\textwidth]{IMAGES/ca2.png}
				\label{fig:ca2}
				\caption{Propagation d'une onde à vitesse constante}
			\end{figure}
		\end{column}
	\end{columns}

\end{frame}

\begin{frame}{\textbf{Résultats et performance}}
    \begin{figure}[H]
		\centering
		\begin{subfigure}[b]{0.24\textwidth}
			\centering
			\includegraphics[width=\textwidth]{IMAGES/rmap1.png}
			\caption*{Résultats carte n°1}
			\label{fig:rmap1}
		\end{subfigure}
		\hfill
		\begin{subfigure}[b]{0.24\textwidth}
			\centering
			\includegraphics[width=\textwidth]{IMAGES/rmap2.png}
			\caption*{Résultats carte n°2}
			\label{fig:rmap2}
		\end{subfigure}
		\hfill
		\begin{subfigure}[b]{0.24\textwidth}
			\centering
			\includegraphics[width=\textwidth]{IMAGES/rmap3.png}
			\caption*{Résultats carte n°3}
			\label{fig:rmap3}
		\end{subfigure}
		\hfill
		\begin{subfigure}[b]{0.24\textwidth}
			\centering
			\includegraphics[width=\textwidth]{IMAGES/rmap4.png}
			\caption*{Résultats carte n°4}
			\label{fig:rmap4}
		\end{subfigure}
		\vfill
		\begin{subfigure}[b]{0.24\textwidth}
			\centering
			\includegraphics[width=\textwidth]{IMAGES/rmap5.png}
			\caption*{Résultats carte n°5}
			\label{fig:rmap5}
		\end{subfigure}
		\hfill
		\begin{subfigure}[b]{0.24\textwidth}
			\centering
			\includegraphics[width=\textwidth]{IMAGES/rmap6.png}
			\caption*{Résultats carte n°6}
			\label{fig:rmap6}
		\end{subfigure}
		\hfill
		\begin{subfigure}[b]{0.24\textwidth}
			\centering
			\includegraphics[width=\textwidth]{IMAGES/rmap7.png}
			\caption*{Résultats carte n°7}
			\label{fig:rmap7}
		\end{subfigure}
		\hfill
		\begin{subfigure}[b]{0.24\textwidth}
			\centering
			\includegraphics[width=\textwidth]{IMAGES/rmap8.png}
			\caption*{Résultats carte n°8}
			\label{fig:rmap8}
		\end{subfigure}
		\caption{Résultats sur les cartes de test}
		\label{fig:results_benchmark_maps}
	\end{figure}
\end{frame}

\begin{frame}{\textbf{Résultats et performance}}
    \begin{table}[H]
		\centering
		\begin{tabular}{c c c c c c}
			\hline
			Carte & Théoriquement & Méthode & Rel. diff. ($\%$) & CA ($\pm 2$ mu) & Rel. Error ($\%$)\\
			\hline
			Carte 1 & 1081 & 1084 & 0.3 & 1086 & 0.5\\
			Carte 2 & 1121 & 1125 & 0.4 & 1122 & 0.1\\
			Carte 3 & 1122 & 1125 & 0.4 & 1122 & 0\\
			Carte 4 & 1084 & 1088 & 0.4 & 1086 & 0.2\\
			Carte 5 & 1521 & 1531 & 0.7 & 1522 & 0.1\\
			Carte 6 & 1166 & - & - & 1168 & 0.2\\
			Carte 7 & 1166 & - & - & 1168 & 0.2\\
			Carte 8 & 1776 & - & - & 1780 & 0.2\\
			\hline
		\end{tabular}
		\caption{Comparaisons des résultats}
		\label{tab:benchmark_results}
	\end{table}
\end{frame}

\begin{frame}{\textbf{Algorithme de Dijkstra}}
	\begin{itemize}
		\item Détection d'impasse :
		\vspace{0.2cm}
		\begin{itemize}
			\item Le robot reste dans un certain périmètre autour de sa position pendant un certain temps.
		\end{itemize}
		\vspace{0.2cm}
		\item En cas d'impasse :
		\vspace{0.2cm}
		\begin{itemize}
			\item Redéfinition du point étape :
			\vspace{0.2cm}
			\begin{itemize}
				\item Plus proche point appartenant aux cellules traversées de l'étape.
			\end{itemize}
			\vspace{0.2cm}
			\item Algorithme de Dijkstra :
			\vspace{0.2cm}
			\begin{itemize}
				\item Existence du chemin garanti.
			\end{itemize}
			\vspace{0.2cm}
			\item Simplification du chemin.
		\end{itemize}
	\end{itemize}

\end{frame}

\begin{frame}{\textbf{Analyses}}
	\begin{itemize}
		\item Portée délimitée d'application
		\vspace{0.2cm}
		\begin{itemize}
			\item Adaptée aux espaces ouverts et cavités moyennes.
			\item En cavités étroites, exécutions fréquentes de Dijkstra.
			\item Implémentation possible en 2D et 3D.
		\end{itemize}
		\vspace{0.2cm}
		\item Robustesse face aux pannes et à l'incertitude
		\vspace{0.2cm}
		\begin{itemize}
			\item Gestion des blocages et sortie de situations difficiles.
			\item Dijkstra assure un retour en arrière en cas de besoin.
		\end{itemize}
		\vspace{0.2cm}
		\item Complétude
		\vspace{0.2cm}
		\begin{itemize}
			\item Exploration de tout le domaine possible.
			\item Nécessité de tests approfondis.
			\item Complétude incertaine à ce stade.
		\end{itemize}
	\end{itemize}

\end{frame}


% Section 3 : Résultats
\section{Communication}

\begin{frame}{\textbf{Communication}}
	\textbf{Objectif}

	\vspace{0.5em}
	
	Coordonner l'exploration de zones inconnues par un essaim de robots, sans entité centrale pour gérer la coordination.
	
	\vspace{0.5em}

	\begin{itemize}
		\item Contraintes
		\begin{itemize}
			\item Environnement confiné
			\vspace{0.2cm}
			\item Liaisons souvent coupées
		\end{itemize}
		\vspace{0.2cm}
		\item Simplification
		\begin{itemize}
			\item Communication possible si contact visuel
			\vspace{0.2cm}
			\item Pas de modélisation d'atténuation
			\vspace{0.2cm}
			\item Pas d'erreur de communication
		\end{itemize}
	\end{itemize}

\end{frame}

\begin{frame}{\textbf{Principe}}
	\begin{columns}
		\begin{column}{0.5\textwidth}
			\begin{itemize}
				\item Sélection dynamique du robot maître :
				\begin{itemize}
					\item Le robot avec le plus petit ID devient le maître.
					\item Si le groupe se divise, un maître est créé.
					\item Si deux groupes se rencontre, les maîtres se synchronisent.
				\end{itemize}
				\item Communication hiérarchique forte, le robot maître :
				\begin{itemize}
					\item donne des instructions aux robots avec un ID plus élevé.
					\item partage toutes les informations acquises par tout le groupe.
				\end{itemize}
				\item Communication robot-à-robot :
				\begin{itemize}
					\item Communication par lumière.
				\end{itemize}
			\end{itemize}
		\end{column}
		\begin{column}{0.5\textwidth}
			\begin{figure}
				\centering
				\includegraphics[width=0.4\textwidth]{IMAGES/color_sensor.png}
				\caption{Exemple de capteur de couleur, ref AS7341 - 10 cannaux}
				\label{fig:color_sensor}
			\end{figure}
		\end{column}
	\end{columns}
\end{frame}

\begin{frame}{\textbf{Principe}}
	\begin{columns}
		\begin{column}{0.5\textwidth}
			\begin{itemize}
				\item Choix des directions de groupe :
				\begin{itemize}
					\item Synchronisation de la carte connue.
					\vspace{0.2cm}
					\item Envoie des information du robot esclave vers le maître.
					\vspace{0.2cm}
					\item Calcul de la table de coût.
					\vspace{0.2cm}
					\item Envoie des directions par le maître en minimisant la combinaison des coût.
				\end{itemize}
			\end{itemize}
		\end{column}
		\begin{column}{0.5\textwidth}
			\begin{table}[H]
				\centering
				\begin{tabular}{c c c c c c}
					\hline
					& \textbf{R. 1} & \textbf{R. 2} & \textbf{R. 3} & \textbf{R. 4} & \textbf{R. 5}\\
					\hline
					\textbf{WP 1} & \textcolor{blue}{$\mathbf{23}$} & $87$ & $234$ & $56$ & $192$ \\
					\textbf{WP 2} & $245$ & $76$ & \textcolor{blue}{$\mathbf{11}$} & $68$ & $39$ \\
					\textbf{WP 3} & $90$ & \textcolor{blue}{$\mathbf{21}$} & $73$ & $\mathbf{50}$ & $164$ \\
					\textbf{WP 4} & $132$ & $58$ & $49$ & $77$ & \textcolor{blue}{$\mathbf{25}$} \\
					\textbf{WP 5} & $181$ & $\mathbf{13}$ & $66$ & \textcolor{blue}{$\mathbf{39}$} & $70$ \\ 
					\hline
				\end{tabular}
				\caption{Exemple de table de coût}
				\label{tab:example_5R5WP}
			\end{table}
			\begin{table}[H]
				\centering
				\begin{tabular}{c c c c c c}
					\hline
					& \textbf{R. 1} & \textbf{R. 2} & \textbf{R. 3} & \textbf{R. 4} & \textbf{R. 5}\\
					\hline
					\textbf{WP 1} & \textcolor{blue}{$\mathbf{61}$} & $125$ & $93$ & $47$ & \textcolor{blue}{$\mathbf{59}$} \\
					\textbf{WP 2} & $88$ & \textcolor{blue}{$\mathbf{12}$} & \textcolor{blue}{$\mathbf{53}$} & \textcolor{blue}{$\mathbf{29}$} & $174$ \\ 
					\hline
				\end{tabular}
				\caption{Exemple de table de coût}
				\label{tab:example_5R2WP}
			\end{table}
		\end{column}
	\end{columns}
\end{frame}


\begin{frame}{\textbf{Encodage}}
	\textbf{Structure d'un message type :}
	$$
	\underbrace{\smallsmile \smallsmile \smallsmile \smallsmile}_{\text{ID émetteur}}~~
	\underbrace{\smallsmile \smallsmile \smallsmile \smallsmile}_{\text{Nb de receveur}}~~
	\underbrace{\smallsmile \smallsmile \smallsmile \smallsmile \dots \smallsmile \smallsmile \smallsmile \smallsmile}_{4 \times \text{Nb de receveur}}~~
	\underbrace{\smallsmile \smallsmile \smallsmile \smallsmile \dots \smallsmile \smallsmile \smallsmile \smallsmile}_{\text{Message}}
	~~~ \cdots
	$$
	\vspace{0.3em}
	$$
	\cdots ~~~
	\underbrace{\smallsmile \smallsmile \smallsmile \smallsmile}_{\text{Indicateur de fin de message}}\;
	\underbrace{\smallsmile \smallsmile \smallsmile \smallsmile}_{\text{Verif. somme}}~~
	\underbrace{\smallsmile \smallsmile \smallsmile \smallsmile}_{\text{ID émetteur avec bit de parité}}
	$$

	\textbf{Resistance aux erreurs par methode ARQ}
\end{frame}


\begin{frame}{\textbf{Table des types}}
    \begin{table}[H]
		\centering
		\begin{tabular}{c l l}
			\hline
			\textbf{Indicateur de type} & \textbf{Type} & \textbf{Taille des données}\\
			\hline
			\texttt{0000} & Fin de message & \texttt{4} bits\\
			\texttt{0001} & Entier & \texttt{8} bits\\
			\texttt{0010} & Entier & \texttt{32} bits\\
			\texttt{0011} & Flottant & \texttt{32} bits\\
			\texttt{0100} & Chaîne de caractères & \texttt{8} bits pour la longueur + \texttt{8} bits par caractère\\
			\texttt{0101} & Liste d'entiers 8 bits & \texttt{16} bits pour la longueur + \texttt{8} bits par entier\\
			\texttt{0110} & Liste de flottants & \texttt{16} bits pour la longueur + \texttt{32} bits par flottant\\
			\texttt{0111} & ID de type de message & \texttt{8} bits\\
			\texttt{1000} & Info robot & \texttt{168} bits\\
			\texttt{1010} & Message binaire & \texttt{16} bits pour la longueur + \texttt{1} bit par bit binaire\\
			\hline
		\end{tabular}
		\caption{Tableau des indicateurs de type}
	\end{table}
\end{frame}

\begin{frame}{\textbf{Table des identifiants de message}}
	\begin{table}[H]
		\centering
		\begin{tabular}{c l}
			\hline
			\textbf{Message ID} & \textbf{Description} \\
			\hline
			\texttt{00000000} & Demander de répéter le dernier message\\
			\texttt{00000001} & Dernier message reçu correctement\\[5pt]

			\texttt{00010000} & Demander l'ensemble des robots connectés\\
			\texttt{00010001} & Envoyer l'ensemble des robots connectés\\[5pt]

			\texttt{00100000} & Demander à tous où ils veulent aller\\
			\texttt{00100001} & Envoyer la position du prochain point de passage avec l'ID de la frontière\\
			\texttt{00100010} & Envoyer les informations du robot\\
			\texttt{00100011} & Envoyer la prochaine position à atteindre\\[5pt]

			\texttt{00110000} & Demander la synchronisation de la carte en direct\\
			\texttt{00110001} & Envoyer la carte en direct mise à jour\\[5pt]

			\texttt{01010101} & Transmettre un message d'un autre robot\\
			\hline
		\end{tabular}
		\caption{Table des identidiants de message}
	\end{table}
\end{frame}

\begin{frame}{\textbf{Analyses}}
	\begin{itemize}
		\item{Portée délimitée d'application}
		\begin{itemize}
			\item Exploration robotique en essaim, scalabilité
			\item Intérieur/extérieur (secours, exploration planétaire, recherche et sauvetage)
			\item Communication optique, propagation lumineuse
		\end{itemize}
		
		\vspace{0.2cm}
		\item{Robustesse face aux pannes et à l'incertitude}
		\begin{itemize}
			\item Detection d'erreurs (ARQ, vérification de somme, bit de parité)
			\item Réaffectation dynamique du rôle de maître
			\item Encodage compact, bande passante réduite
		\end{itemize}
		
		\vspace{0.2cm}
		\item{Vitesse}
		\begin{itemize}
			\item Encodage optimisé
			\item Cycle requête-réponse, réduire retransmissions
			\item Communication parallèle, traitement simultané
		\end{itemize}
	\end{itemize}
\end{frame}

% Section 4 : Discussions
\section{Simulateur et résultats}

\begin{frame}{\textbf{Simulateur}}
	\textbf{Objectif}

	\vspace{0.5em}

	Implementer et tester les algorithmes proposés pour les améliorer avant un déployement sur robot
	
	\vspace{0.5em}

	\begin{itemize}
		\item Contraintes
		\begin{itemize}
			\item Simulateur en temps réel
			\vspace{0.2cm}
			\item Contrainte cinématique du robot prise en considération
		\end{itemize}
		\vspace{0.2cm}
		\item Simplification
		\begin{itemize}
			\item Les robots opèrent dans les airs et se déplaçant sur le sol
			\vspace{0.2cm}
			\item Le sol est approximé à un plan 2D
			\vspace{0.2cm}
			\item Aucune irrégularité de surface n'est prise en compte
		\end{itemize}
	\end{itemize}
\end{frame}

\begin{frame}{\textbf{Génération de la carte}}
	\begin{columns}[t]
		\begin{column}{0.33\textwidth}

			\vspace{1em}
			Génération des obstacles par automate cellulaire sur grille cartesienne
			\vspace{1em}
			
			\textbf{Règles :}

			\vspace{1em}

			Si 4 cellules sont actives dans le voisinage de Moore alors la cellule s'active sinon elle se déactive
		\end{column}

		\begin{column}{0.33\textwidth}
			\begin{figure}
				\centering
				\includegraphics[width=0.9\textwidth]{IMAGES/grid_transform_map_generation.png}
				\caption{Transformation de la grille}
				\label{fig:map_generation_step3}
			\end{figure}
			\begin{figure}
				\centering
				\includegraphics[width=0.9\textwidth]{IMAGES/marching_square_triangle.png}
				\caption{Méthode de Marching Square appliquée aux triangles}
				\label{fig:map_generation_step1}
			\end{figure}
		\end{column}
		\begin{column}{0.33\textwidth}
			\begin{figure}
				\centering
				\includegraphics[width=0.8\textwidth]{IMAGES/map_generation_example.png}
				\caption{Exemple de carte générée}
				\label{fig:map_generation_step2}
			\end{figure}
		\end{column}
	\end{columns}
\end{frame}

\begin{frame}{\textbf{Exemple de cartes générées}}
    \begin{figure}[H]
		\centering
		\begin{subfigure}{0.32\textwidth}
			\centering
			\includegraphics[width=\textwidth]{IMAGES/map_dx100.png}
			\caption{$\delta x = 100$ mu}
			\label{fig:real_map_100}
		\end{subfigure}
		\hfill
		\begin{subfigure}{0.32\textwidth}
			\centering
			\includegraphics[width=\textwidth]{IMAGES/map_dx40.png}
			\caption{$\delta x = 40$ mu}
			\label{fig:real_map_40}
		\end{subfigure}
		\hfill
		\begin{subfigure}{0.32\textwidth}
			\centering
			\includegraphics[width=\textwidth]{IMAGES/map_dx10.png}
			\caption{$\delta x = 10$ mu}
			\label{fig:real_map_10}
		\end{subfigure}
		\caption{Exemple de cartes générées}
		\label{fig:three_map_example}
	\end{figure}
\end{frame}

\begin{frame}{\textbf{Implémentation des capteurs}}
	\begin{columns}
		\begin{column}{0.5\textwidth}
			\begin{itemize}
				\item Centrale inertielle
				\vspace{0.2cm}
				\item LiDAR
				\begin{itemize}
					\item Émission
					\item Réflexion
					\item Réception
					\item Calcul du temps de vol $\rightarrow$ distance.
				\end{itemize}
				\vspace{0.2cm}
				\item Optimisation
				\begin{itemize}
					\item Segmentation de l'espace
					\item Algorithme de Bresenham modifié
				\end{itemize}
				\item Simulation des erreurs par loi normale
			\end{itemize}
		\end{column}
		\begin{column}{0.5\textwidth}
			\begin{figure}
				\centering
				\includegraphics[width=0.6\textwidth]{IMAGES/lidar_operation_scheme.png}
				\caption{Fonctionnement d'un LiDAR}
				\label{fig:lidar_ope}
			\end{figure}
			\begin{figure}
				\centering
				\includegraphics[width=0.6\textwidth]{IMAGES/draw_line.png}
				\caption{Algorithme de Bresenham modifié}
				\label{fig:bresenham_algorithm}
			\end{figure}
			
		\end{column}
	\end{columns}
\end{frame}


\begin{frame}{\textbf{localisation et cartographie simultanées (SLAM)}}
	\begin{columns}[t]
		\begin{column}{0.5\textwidth}
			\begin{itemize}
				\item Création d'une carte de caractéristique avec 4 états:
				\begin{itemize}
					\item Inconnue (Gris)
					\item Obstacle (Violet)
					\item Libre (Vert clair)
					\item Traversé (Vert)
				\end{itemize}				
			\end{itemize}
		\end{column}
		\begin{column}{0.5\textwidth}
			\begin{itemize}
			\item Calcul et correction de la position
				\begin{itemize}
					\item Centrale inertielle
					\item Triangulation par LiDAR
				\end{itemize}
			\end{itemize}
		\end{column}
	\end{columns}
	\begin{columns}[t]
		\begin{column}{0.5\textwidth}
			\begin{figure}
				\centering
				\includegraphics[width=0.5\textwidth]{IMAGES/lgm_t3_c.png}
				\caption{Carte des caractéristiques}
			\end{figure}
		\end{column}
		\begin{column}{0.5\textwidth}
			\begin{figure}
				\centering
				\includegraphics[width=0.6\textwidth]{IMAGES/lidar_correct_pos.png}
				\caption{Correction de la position par le LiDAR}
			\end{figure}
		\end{column}
	\end{columns}
		
\end{frame}

\begin{frame}{\textbf{Carte de caractéristique dynamique}}
	\begin{columns}
		\begin{column}{0.7\textwidth}
			Environnement hautement dynamique
			\begin{itemize}
				\item Carte jumelle : carte d'occurrence
				\item Compte le nombre de rayons de LiDAR ayant touché les cellules de la carte de caractéristiques
				\begin{itemize}
					\item Si collision avec un mur : Ajoute 3
					\item Sinon : Soustraire 1
				\end{itemize}
			\end{itemize}
			\vspace{0.5em}
			
			Assure une carte des caractéristiques sécurisante tout en supprimant de celle-ci les objets dynamiques

			\begin{figure}[H]
				\centering
				\begin{subfigure}{0.32\textwidth}
					\centering
					\includegraphics[width=0.6\textwidth]{IMAGES/lgm_t1_c.png}
				\end{subfigure}
				\hfill
				\begin{subfigure}{0.32\textwidth}
					\centering
					\includegraphics[width=0.6\textwidth]{IMAGES/lgm_t2_c.png}
				\end{subfigure}
				\hfill
				\begin{subfigure}{0.32\textwidth}
					\centering
					\includegraphics[width=0.6\textwidth]{IMAGES/lgm_t3_c.png}
				\end{subfigure}
			\end{figure}
			

			Amélioration possible : Filtre de Kalman
		\end{column}
		\begin{column}{0.3\textwidth}
			\begin{figure}
				\centering
				\includegraphics[width=0.6\textwidth]{IMAGES/lidar_occurance_map.png}
				\caption{Carte dynamique des caractéristiques}
			\end{figure}
		\end{column}
	\end{columns}

\end{frame}


\begin{frame}{\textbf{Résultats}}
	\begin{columns}
		\begin{column}{0.8\textwidth}
			\begin{table}[H]
				\centering
				\begin{tabular}{c c c c}
					\hline
					\textbf{Carte n°} & \textbf{Nb Robots} & \textbf{Temps exploration} & \textbf{Réduction} \\ 
					\hline
					30 & 1 & 2'20" & - \\
					30 & 2 & 1'3" & 55\% \\
					40 & 1 & 2'35" & - \\
					40 & 2 & 1'32" & 41\% \\
					55 & 1 & 2'7" & - \\
					55 & 2 & 54" & 57\% \\
					100 & 1 & 3'16" & - \\
					100 & 3 & 1'28" & 55\% \\
					100 & 5 & 1'12" & 63\% \\
				\end{tabular}
				\caption{Temps d'exploration pour différentes tailles de carte et nombre de robots}
				\label{tab:exploration_times_1}
			\end{table}
		\end{column}
		\begin{column}{0.2\textwidth}
			\begin{figure}[H]
				\centering
				\begin{subfigure}[b]{0.8\textwidth}
					\centering
					\includegraphics[width=\textwidth]{IMAGES/map30.png}
				\end{subfigure}
				\vfill
				\begin{subfigure}[b]{0.8\textwidth}
					\centering
					\includegraphics[width=\textwidth]{IMAGES/map40.png}
				\end{subfigure}
				\vfill
				\begin{subfigure}[b]{0.8\textwidth}
					\centering
					\includegraphics[width=\textwidth]{IMAGES/map55.png}
				\end{subfigure}
				\vfill
				\begin{subfigure}[b]{0.8\textwidth}
					\centering
					\includegraphics[width=\textwidth]{IMAGES/map100.png}
				\end{subfigure}
			\end{figure}
		\end{column}
	\end{columns}
\end{frame}

\begin{frame}{\textbf{Résultats}}
	\begin{columns}
		\begin{column}{0.33\textwidth}
			\begin{figure}[H]
				\centering
				\includegraphics[width=0.9\textwidth]{IMAGES/map55_explored_2robot.png}
				\caption*{Carte n°55 - 2 robots}
			\end{figure}
		\end{column}
		\begin{column}{0.33\textwidth}
			\begin{figure}[H]
				\centering
				\begin{subfigure}{0.9\textwidth}
					\centering
					\includegraphics[width=\textwidth]{IMAGES/map40_explored_1robot.png}
					\caption*{Carte n°40 - 1 robot}
				\end{subfigure}
				\vfill
				\begin{subfigure}{0.9\textwidth}
					\centering
					\includegraphics[width=\textwidth]{IMAGES/map40_explored_2robot.png}
					\caption*{Carte n°40 - 2 robots}
				\end{subfigure}
			\end{figure}
		\end{column}
		\begin{column}{0.33\textwidth}
			\begin{figure}[H]
				\centering
				\includegraphics[width=0.9\textwidth]{IMAGES/map100_explored_5robot.png}
				\caption*{Carte n°100 - 5 robots}
			\end{figure}
		\end{column}
	\end{columns}
\end{frame}

% Section 5 : Conclusion
\section{Conclusion et perspectives}
\begin{frame}{\textbf{Conclusion}}
	\begin{itemize}
		\item Développement d'un simulateur :
		\begin{itemize}
			\item Environ 3500 lignes de code
			\item Une dizaine de fonctions utiles pour de futures implémentations
		\end{itemize}
		\vspace{0.2cm} 
		\item Validation des algorithmes d'exploration dans des environnements complexes
		\vspace{0.2cm} 
		\item Mise en place d'une communication simple mais robuste entre les robots
		\vspace{0.2cm} 
		\item Première étape vers des recherches approfondies en thèse
	\end{itemize}

\end{frame}


\begin{frame}{\textbf{Perspectives}}
	\begin{itemize}
		\item Amélioration de la performance du simulateur :
		\begin{itemize}
			\item Passage en C++
			\item Parallélisation
		\end{itemize}
		\vspace{0.2cm}
		\item Implémentation ROS
		\vspace{0.2cm}
		\item Implémentation du terrain
	\end{itemize}


\end{frame}

\begin{frame}{\textbf{Perspectives}}
    \begin{columns}[t]
        \begin{column}{0.55\textwidth}
            \textbf{Simplification du problème d'optimisation} \\
            \vspace{0.5em}
            \begin{itemize}
                \item Minimiser la consommation d'énergie lors du déplacement de A à B.
                \vspace{0.2cm}
                \item Consommation d'énergie : $\displaystyle E(T) = \int_{0}^{T} \| \omega(t) \| \, dt$
                \vspace{0.2cm}
                \item Où : $\| \omega(t) \| = \sqrt{\omega_L^{2}(t) + \omega_R^{2}(t)}$
                \vspace{0.2cm}
                \item Contraintes : \( X(0) = X_R \) et \( X(T) = X_{WP} \).
                \vspace{0.2cm}
                \item Problématique : Assurer \( X(T) = X_{WP} \).
            \end{itemize}
        \end{column}

        \begin{column}{0.45\textwidth}
			\begin{figure}[H]
				\centering
				\begin{subfigure}{0.45\textwidth}
					\centering
					\includegraphics[width=\textwidth]{IMAGES/robot_model_2.png}
				\end{subfigure}
				\hfill
				\begin{subfigure}{0.45\textwidth}
					\centering
					\includegraphics[width=\textwidth]{IMAGES/robot_3Dmodel.png}
				\end{subfigure}
				\caption{Modèle de robot utilisé}
			\end{figure}

            \begin{figure}[H]
                \centering
                \begin{subfigure}[b]{0.48\textwidth}
                    \includegraphics[width=\textwidth]{IMAGES/random.png}
                    \caption{Chemin stochastique}
                \end{subfigure}
                \hfill
                \begin{subfigure}[b]{0.48\textwidth}
                    \includegraphics[width=\textwidth]{IMAGES/empiric.png}
                    \caption{Chemin empirique}
                \end{subfigure}
                \caption{Comparaison des approches d'optimisation des chemins stochastiques et empiriques}
            \end{figure}
        \end{column}
    \end{columns}
\end{frame}

\begin{frame}{\textbf{Références}}
	\scriptsize
	\begin{thebibliography}{9}

	\bibitem{tapia_2016}
	Calvo Tapia, Carlos, Villacorta-Atienza, José, Mironov, Vasily, Gallego, Victor, and Makarov, Valeri. (2016). WAVES in ISOTROPIC TOTALISTIC CELLULAR AUTOMATA: APPLICATION to REAL-TIME ROBOT NAVIGATION. \textit{Advances in Complex Systems}, 19, 1650012. DOI 10.1142/S0219525916500120.

	\end{thebibliography}

\end{frame}

\end{document}
