\documentclass[../main.tex]{subfiles}
\LoadClass[a4paper,12pt]{article}
\documentclass{article}

\usepackage[english]{babel}

%%%%%%%%%%%%%%%%%%%%%%%%%%%%%%%
%     import des packages     %
%%%%%%%%%%%%%%%%%%%%%%%%%%%%%%%
\usepackage[export]{adjustbox}
\usepackage{algorithm}
\usepackage{algorithmic}
\usepackage{amsmath,amsfonts,amssymb}
\usepackage{anyfontsize}
\usepackage{array}
\usepackage[french]{babel}
\usepackage{colortbl}
\usepackage{comment}
\usepackage{cclicenses}
\usepackage{eqnarray}
\usepackage{eso-pic}
\usepackage{dirtree}
\usepackage{fancybox}
\usepackage{fancyhdr}
\usepackage{float}
\usepackage[T1]{fontenc} 
\usepackage{forest}
\usepackage{fourier-orns}
\usepackage{gensymb}
\usepackage{geometry}
\usepackage{glossaries}
\usepackage{graphicx}
\usepackage{hyperref}
\usepackage{ifthen}
\usepackage{import}
\usepackage{indentfirst}
\usepackage[utf8]{inputenc}
\usepackage{lastpage}
\usepackage{libertine}
\usepackage{lipsum}
\usepackage{listings}
\usepackage{mathtools}
\usepackage{mdframed}
\usepackage{multicol}
\usepackage{pdfpages}
\usepackage{pifont}
\usepackage{stmaryrd}
\usepackage{subcaption}
\usepackage{subfiles}
\usepackage{tabularx}
% \usepackage{tcolorbox}
\usepackage[most]{tcolorbox}
\usepackage{textcomp}
\usepackage{ulem}
\usepackage{wrapfig}

%%%%%%%%%%%%%%%%%%%%%%%%%%%%%%%%%%%%%%%%%%%%%%%%%%%%%%%%%
%    Renseigner les titres et variables importantes     %
%%%%%%%%%%%%%%%%%%%%%%%%%%%%%%%%%%%%%%%%%%%%%%%%%%%%%%%%%
\newcommand{\titre}{Multi-robot coordination}
\newcommand{\soustitre}{Autonomous exploration of gallery networks}
\newcommand{\sujet}{Engineering Graduation Project}
\newcommand{\sujets}{Seatech 3A - MOCA}
\newcommand{\auteur}{Fabien MATHÉ}
\newcommand{\referent}{M. Mehmet ERSOY}
\newcommand{\reportdate}{\date}

\newcommand{\partA}{State of the art}
\newcommand{\partB}{Notations}
\newcommand{\partC}{Path planning}
\newcommand{\partD}{Communication}
\newcommand{\partE}{Simulator implementation}

%%%%%%%%%%%%%%%%%%%
%     BOOLEEN     %
%%%%%%%%%%%%%%%%%%%

% Renseigner si le Rapport contient un abstract
\setboolean{abst}{true}
% Renseigner si le Rapport contient des remerciements
\setboolean{thx}{true}
% Renseigner si le Rapport contient une table des matières
\setboolean{contents}{true}
% Renseigner si le Rapport contient une introduction
\setboolean{introduction}{true}
% Renseigner si le Rapport contient une partie 2
\setboolean{pt2}{true}
% Renseigner si le Rapport contient une partie 3
\setboolean{pt3}{true}
% Renseigner si le Rapport contient une partie 4
\setboolean{pt4}{true}
% Renseigner si le Rapport contient une partie 5
\setboolean{pt5}{true}
% Renseigner si le Rapport contient une introduction
\setboolean{conclusion}{true}
% Renseigner si le Rapport contient des perspectives
\setboolean{perspectives}{true}
% Renseigner si le document contient une bibliographie
\setboolean{biblio}{true} 
% Renseigner si le document contient un glossaire
\setboolean{glossaire}{false}
% Renseigner si le Rapport contient des annexes 
\setboolean{annexe}{true}


%%%%%%%%%%%%%%%%%%%%%%%%%%%%%%%%%%%%%%
%     En-têtes en pieds de pages     %
%%%%%%%%%%%%%%%%%%%%%%%%%%%%%%%%%%%%%%
\geometry{hmargin=2cm,vmargin=2.3cm}
\pagestyle{fancy}
\fancyhfoffset[]{0pt}
\setlength{\headheight}{28pt}
\lhead{\includegraphics[height = 0.6cm]{IMAGES/logos/Logo_SeaTech_2023.png}}
% \rhead{\includegraphics[height = 0.7cm]{IMAGES/logos/MOCA.png}}
\rhead{\textsc{\leftmark}}

% Update \rightmark with \section name
\renewcommand{\sectionmark}[1]{\markboth{#1}{#1}}


\lfoot{\auteur}
\cfoot{ }
\rfoot{Page \thepage \ / \pageref{LastPage}}

\title{\titre}
\author{\auteur}
\date{\today}

%%%%%%%%%%%%%%%%%%%%%%%%%%%%%%
%     Autre mise en page     %
%%%%%%%%%%%%%%%%%%%%%%%%%%%%%%
\numberwithin{figure}{section}
\numberwithin{table}{section}

\setcounter{tocdepth}{2} % Change to 1 to exclude subsections as well


\newcommand{\citeURL}[1]{\href{#1}{\detokenize{#1}}}

% Création du compteur d'annexes
\newcounter{annexecounter}

% Définition de la commande pour les annexes
\NewDocumentCommand{\annexe}{m}{%
    \stepcounter{annexecounter} % Incrémenter le compteur d'annexes
    \subsection*{Annexe \arabic{annexecounter} - #1} % Affichage du texte avec le numéro et le titre
	\label{sec:#1}
}

\newcommand{\tobedone}{\textcolor{red}{\LARGE \textbf{TO BE DONE}}}
\newcommand{\annexetonum}{\textcolor{red}{\LARGE \textbf{ANNEXE ...}}}
\newcommand{\figtonum}{\textcolor{red}{\textbf{FIGURE ...}}}

\renewcommand{\familydefault}{\sfdefault}



%%%%%%%%%%%%%%%%%%%%%%%%%%%%%%%%%
%     Mise en page des codes    %
%%%%%%%%%%%%%%%%%%%%%%%%%%%%%%%%%
\input{ANNEXES/codes/display/python_code.tex}
\input{ANNEXES/codes/display/cpp_code.tex}
\input{ANNEXES/codes/display/f90_code.tex}


%%%%%%%%%%%%%%%%%%%%%%%%%%%%%%%%%%%%%%%%%%%%%%%%%%%%%%%%%%%%%%%%%%%%%%%%%%%%%%%%%%%%%%%%%%%%%%%%%%%%%%%%%%%%%%%%%%%%%%%
%                                                  Début du document                                                  %
%%%%%%%%%%%%%%%%%%%%%%%%%%%%%%%%%%%%%%%%%%%%%%%%%%%%%%%%%%%%%%%%%%%%%%%%%%%%%%%%%%%%%%%%%%%%%%%%%%%%%%%%%%%%%%%%%%%%%%%

\begin{document}

%%%%%%%%%%%%%%%%%%%%%%%%%
%     Page de garde     %
%%%%%%%%%%%%%%%%%%%%%%%%%
\begin{titlepage}
	\AddToShipoutPictureBG*{\includegraphics[width=\paperwidth,height=\paperheight]{IMAGES/PageDeGardeRapport.png}}
	\begin{figure}[H]
		\begin{subfigure}{0.45\linewidth}
				\includegraphics[width=0.6\textwidth,left]{IMAGES/logos/Logo_SeaTech_2023.png}
		\end{subfigure}
		\hfill
		\begin{subfigure}{0.45\linewidth}
				% \includegraphics[width=0.6\textwidth,right]{IMAGES/logos/MOCA.png}
		\end{subfigure}
	\end{figure}

	\centering

	% Espacement vertical
	\vspace*{5cm}

	% Barres horizontales
	\makebox[0.7\linewidth]{\hrulefill}\\[0.2cm]

	% Titre encadré
	\vspace{0.5cm}
	\begin{minipage}{\textwidth}
		\centering
		{\fontsize{28}{48}\selectfont \textsc{\titre}}\\[0.2cm]

		{\fontsize{18}{48}\selectfont \textsc{\soustitre}}
	\end{minipage}
	\vspace{0.3cm}

	% Barres horizontales
	\makebox[0.8\linewidth]{\hrulefill}\\[0.2cm]

	% Espacement vertical
	\vspace{3cm}

	% Description
		\large{\Large \textbf{\sujet}}\\
		\large{\textbf{\sujets}}\\

		\vspace{0.5cm}
		\large{\textbf{\reportdate}}

	\vspace{2cm}

	\begin{minipage}{0.20\textwidth}

	\end{minipage}
	\hfill
	\begin{minipage}{0.35\textwidth}
		\begin{flushleft}
			Auteur : \\
			\auteur
		\end{flushleft}
	\end{minipage}
	\begin{minipage}{0.09\textwidth}
		% Section vide pour espacement optimal
	\end{minipage}
	\hfill
 	\begin{minipage}{0.3\textwidth}
		\begin{flushleft}
			Enseignant : \\
			\referent

		\end{flushleft}
	\end{minipage}


\end{titlepage}

\ClearShipoutPictureBG

\newpage

\renewcommand{\thepage}{}

\renewcommand{\thepage}{\arabic{page}}
\renewcommand{\thesection}{\Roman{section}}

%%%%%%%%%%%%%%%%%%
%     Résumé     %
%%%%%%%%%%%%%%%%%%
\ifthenelse{\boolean{abst}}{
	\addcontentsline{toc}{section}{\protect\numberline{}Résumé}%
	\subfile{SECTIONS/1resume}

	\newpage
}

%%%%%%%%%%%%%%%%%%%%%%%%%
%     Remerciements     %
%%%%%%%%%%%%%%%%%%%%%%%%%
\ifthenelse{\boolean{thx}}{
	\addcontentsline{toc}{section}{\protect\numberline{}Remerciements}%
	\subfile{SECTIONS/2remerciements}

	\newpage
}

%%%%%%%%%%%%%%%%%%%%%%%%%%%%
%     Plan du document     %
%%%%%%%%%%%%%%%%%%%%%%%%%%%%

\ifthenelse{\boolean{contents}}{
	\vfill
	\tableofcontents
	\vfill
	
	\newpage
}

%%%%%%%%%%%%%%%%%%%%%%%%
%     INTRODUCTION     %
%%%%%%%%%%%%%%%%%%%%%%%%
\ifthenelse{\boolean{introduction}}
{
	\addcontentsline{toc}{section}{\protect\numberline{}Introduction}%
	\section*{Introduction}

	\markboth{Introduction}{Introduction} % Manually update \rightmark for section*
	\subfile{SECTIONS/3introduction}


	\newpage
}


%%% PARTIE 1 %%%
\section{\partA}
\subfile{SECTIONS/part1}

%%% PARTIE 2 %%%
\newpage
\ifthenelse{\boolean{pt2}}
{
	\section{\partB}
	\subfile{SECTIONS/part2}
	
	\newpage
}
	
	
%%% PARTIE 3 %%%
\ifthenelse{\boolean{pt3}}
{
	\section{\partC}
	\subfile{SECTIONS/part3}
	
	\newpage
}
	
	
%%% PARTIE 4 %%%
\ifthenelse{\boolean{pt4}}
{
	\section{\partD}
	\subfile{SECTIONS/part4}
	
	\newpage
}
	
%%% PARTIE 5 %%%
\ifthenelse{\boolean{pt5}}{
	\section{\partE}
	\subfile{SECTIONS/part5}

	\newpage
}
		
%%%%%%%%%%%%%%%%%%%%%%
%     CONCLUSION     %
%%%%%%%%%%%%%%%%%%%%%%
\ifthenelse{\boolean{conclusion}}
{
	\addcontentsline{toc}{section}{\protect\numberline{}Conclusion}%
	\section*{Conclusion}
	\markboth{Conclusion}{Conclusion} % Manually update \rightmark for section*
	
	\subfile{SECTIONS/Wconclusion}

	\newpage
}



\ifthenelse{\boolean{perspectives}}
{
	\section*{Perspectives}
	\markboth{Perspectives}{Perspectives} % Manually update \rightmark for section*
	\addcontentsline{toc}{section}{\protect\numberline{}Perspectives}
	\subfile{SECTIONS/Xperspectives}
	
	\newpage 
}

%%%%%%%%%%%%%%%%%%%%%%%%%
%     Bibliographie     %
%%%%%%%%%%%%%%%%%%%%%%%%%

\ifthenelse{\boolean{biblio}}
{
	\addcontentsline{toc}{section}{\protect\numberline{}References}
	% \bibliographystyle{unsrt}
	\bibliographystyle{IEEEtran}
	\footnotesize{\bibliography{BIBLIOGRAPHY/bib.bib}}

	\newpage
}


%%%%%%%%%%%%%%%%%%%%%
%     Glossaire     %
%%%%%%%%%%%%%%%%%%%%%
\normalsize
\ifthenelse{\boolean{glossaire}}
{
	\section*{Glossaire}
	\makeglossaries
	\printglossaries
	\addcontentsline{toc}{section}{\protect\numberline{}Glossaire}%
	\subfile{SECTIONS/Yglossaire}
	
	\newpage
}

%%%%%%%%%%%%%%%%%%%
%     Annexes     %
%%%%%%%%%%%%%%%%%%%
\ifthenelse{\boolean{annexe}}
{
	\section*{Annexes}
	\addcontentsline{toc}{section}{\protect\numberline{}Annexes}%
	\subfile{SECTIONS/Zannexes}
}


\end{document}


\begin{document}

One of the key reference works used in this study is S. M. LaValle's book, Planning Algorithms \cite{Lavalle_2006}, for which I am particularly grateful for his commitment to making his work accessible to the public ({\scriptsize \url{https://lavalle.pl/}}).

\subsection{Multi-Robot Exploration of Cavities}

The exploration of cavities is essential for underground mapping, scientific research, and emergency interventions. However, it poses unique challenges for autonomous robotics.\cite{nationalgeographic_greenland_caves_2025, scalea_2019} Multi-robot exploration of such environments is a growing research field with applications in various domains, including space exploration, cave exploration, urban search and rescue missions, and mining operations.\cite{dang_2021,kambesis_2007} These environments are often very complex, presenting significant challenges for mission planning, such as unpredictable obstacles, inaccessible areas for mobile robots, and the absence of GPS signals. Multi-robot cooperation can help overcome these difficulties by leveraging distributed sensing and coordinated decision-making.

\subsubsection{Challenges in Multi-Robot Cave Exploration}

Exploring cavities with multiple robots involves several technical challenges. Two fundamental aspects are managing cooperation between robots and handling information in complex environments. The technical difficulties encountered are numerous.\cite{scalea_2019}

\vspace{1em}

A particularly successful research field of multi-robot exploration has been in extraterrestrial environments. On earth, the DARPA Subterranean Challenge, launched in 2019, tested autonomous robot teams in complex underground environments, highlighting challenges related to mapping, navigation, and search under severe communication constraints. This challenge further increased interest in autonomous supervision approaches for groups of robots operating in tunnels and mines.\cite{otsu_2020} 

\vspace{1em}

Several robotic approaches have been developed to tackle these challenges. Some methods involve hybrid robotic teams consisting of wheeled or legged robots combined with aerial drones.\cite{dang_2020} In other cases, solely using flying drones has proven sufficient for exploring and mapping cavities.\cite{dharmadhikari_2021, zhang_2017, husian_2013, petracek_2021}

\subsubsection{Underwater Cave Exploration}

Underwater cave exploration is another area of active research. Water provides a stable medium, reducing the difficulty of maintaining position compared to aerial exploration. However, underwater exploration remains highly challenging due to complex communication constraints and the difficulty of localizing robots relative to external references. Ongoing research focuses on semantic segmentation for underwater cave exploration to enhance perception and navigation capabilities.\cite{gupta_2025, abdullah_2024}



\subsection{Path Planning for a Mechatronic System}

Path planning for a mechatronic system is the foundation of all autonomous mobile systems, whether they involve drones or handling and assembly arms. The principle of path or trajectory planning is to determine a solution —a path or a trajectory— connecting a starting point to a target point.

\vspace{1em}

The distinction between path planning and trajectory planning lies in the fact that a planned path is not necessarily feasible for a robot. Indeed, path planning does not always take into account the physical or kinematic constraints of the mobile robot.

\vspace{1em}

Obstacle avoidance is an essential constraint in both approaches. Obstacles define inaccessible areas, and as we will see later, their nature—mobile or static—largely determines the method to be used to solve the planning problem.\\

\vspace{1em}

Among deterministic methods, there is a wide variety of approaches mainly based on three different strategies: graph-based methods, cellular decomposition methods, and potential field methods \cite{latombe_robot_1991,bhattacharyya_robot_2008}.

\vspace{1em}

Graph-based methods consist of constructing a map of feasible paths by considering the obstacles in the scene. Among these graph-based methods, four different types can be distinguished: visibility graphs \cite{visibility_graph_1979}, Voronoi diagrams \cite{garrido_path_2006}, and the Silhouette method \cite{bhattacharyya_robot_2008}.

\vspace{1em}

Methods associated with cellular decomposition consist of dividing the robot's free space into simple regions, called cells, where it is easy to generate a path between two configurations. A graph representing the adjacency relationships between the cells is then constructed and explored \cite{latombe_robot_1991,zhu_new_1991,kedem_efficient_1990,avnaim_practical_1988}.

\vspace{1em}

Another method relies on a fine subdivision of the space to identify free zones. The potential field method is based on this idea by defining potentials that represent attractive forces directed toward the target coordinates and repulsive forces corresponding, for example, to obstacles. The path is then determined by following the opposite of the gradient of the total computed potential \cite{latombe_robot_1991,koren_potential_1991}.

\vspace{1em}

Alternative approaches were developed in the 1990s and 2000s, particularly stochastic path planning methods such as Random Path Planners (RPP) and Probabilistic Roadmap Planners (PRM). These methods involve randomly sampling the space to create a graph of possible paths (\textit{roadmap}) within this space \cite{amato_1996,hsu_2002,nissoux_1999}.

\vspace{1em}

One of the main advantages of this path planning method is that it does not depend on the structure of the space, the number of obstacles, or their arrangement. Once the \textit{roadmap} is created, it is sufficient to use a graph-based path search method to find the path leading to the target point \cite{gasparetto_2015}.


\subsection{Inter-Robot Communication in Cave Environments}

Inter-robot communication (IRC) in cave environments poses challenges due to confined spaces, obstacles, and signal disruptions. IRC methods vary based on technology and communication strategies.

\vspace{1em}

Traditional approaches rely on wireless networks, such as radio frequency (RF) communication, but suffer from signal attenuation in caves. Ad hoc methods, like mesh networks, allow robots to relay data dynamically, extending coverage despite obstacles \cite{silva_2022}. The Autonomous and Collaborative High-Bandwidth Operations with Radio Droppables (ACHORD) framework enhances multi-robot coordination by adapting to intermittent connectivity \cite{silva_2022}.

\vspace{1em}

Acoustic and ultrasonic communication offer robust short-range alternatives, while optical methods (visible light or infrared) provide interference-resistant, high-precision transmissions \cite{klaesson_2020}. Techniques like real-time signal prediction improve communication-aware exploration \cite{clark_2022}.

\vspace{1em}

Cooperative protocols optimize information flow through adaptive routing and dynamic network partitioning. Opportunistic communication, based on intermittent data exchanges, is particularly useful in environments with connectivity constraints \cite{banfi_2018}. Large-scale cave exploration studies highlight the importance of maintaining reliable links for coordination \cite{petracek_2021}.

\vspace{1em}

A major challenge remains optimizing bandwidth, resource management, and interference control. Hybrid systems combining multiple technologies enhance resilience in dynamic environments \cite{klaesson_2020}. The ACHORD framework integrates network design with high-level decision-making to improve subterranean communication \cite{silva_2022}.




\end{document}