\documentclass[../main.tex]{subfiles}
\LoadClass[a4paper,12pt]{article}
\documentclass{article}

\usepackage[english]{babel}

%%%%%%%%%%%%%%%%%%%%%%%%%%%%%%%
%     import des packages     %
%%%%%%%%%%%%%%%%%%%%%%%%%%%%%%%
\usepackage[export]{adjustbox}
\usepackage{algorithm}
\usepackage{algorithmic}
\usepackage{amsmath,amsfonts,amssymb}
\usepackage{anyfontsize}
\usepackage{array}
\usepackage[french]{babel}
\usepackage{colortbl}
\usepackage{comment}
\usepackage{cclicenses}
\usepackage{eqnarray}
\usepackage{eso-pic}
\usepackage{dirtree}
\usepackage{fancybox}
\usepackage{fancyhdr}
\usepackage{float}
\usepackage[T1]{fontenc} 
\usepackage{forest}
\usepackage{fourier-orns}
\usepackage{gensymb}
\usepackage{geometry}
\usepackage{glossaries}
\usepackage{graphicx}
\usepackage{hyperref}
\usepackage{ifthen}
\usepackage{import}
\usepackage{indentfirst}
\usepackage[utf8]{inputenc}
\usepackage{lastpage}
\usepackage{libertine}
\usepackage{lipsum}
\usepackage{listings}
\usepackage{mathtools}
\usepackage{mdframed}
\usepackage{multicol}
\usepackage{pdfpages}
\usepackage{pifont}
\usepackage{stmaryrd}
\usepackage{subcaption}
\usepackage{subfiles}
\usepackage{tabularx}
% \usepackage{tcolorbox}
\usepackage[most]{tcolorbox}
\usepackage{textcomp}
\usepackage{ulem}
\usepackage{wrapfig}

%%%%%%%%%%%%%%%%%%%%%%%%%%%%%%%%%%%%%%%%%%%%%%%%%%%%%%%%%
%    Renseigner les titres et variables importantes     %
%%%%%%%%%%%%%%%%%%%%%%%%%%%%%%%%%%%%%%%%%%%%%%%%%%%%%%%%%
\newcommand{\titre}{Multi-robot coordination}
\newcommand{\soustitre}{Autonomous exploration of gallery networks}
\newcommand{\sujet}{Engineering Graduation Project}
\newcommand{\sujets}{Seatech 3A - MOCA}
\newcommand{\auteur}{Fabien MATHÉ}
\newcommand{\referent}{M. Mehmet ERSOY}
\newcommand{\reportdate}{\date}

\newcommand{\partA}{State of the art}
\newcommand{\partB}{Notations}
\newcommand{\partC}{Path planning}
\newcommand{\partD}{Communication}
\newcommand{\partE}{Simulator implementation}

%%%%%%%%%%%%%%%%%%%
%     BOOLEEN     %
%%%%%%%%%%%%%%%%%%%

% Renseigner si le Rapport contient un abstract
\setboolean{abst}{true}
% Renseigner si le Rapport contient des remerciements
\setboolean{thx}{true}
% Renseigner si le Rapport contient une table des matières
\setboolean{contents}{true}
% Renseigner si le Rapport contient une introduction
\setboolean{introduction}{true}
% Renseigner si le Rapport contient une partie 2
\setboolean{pt2}{true}
% Renseigner si le Rapport contient une partie 3
\setboolean{pt3}{true}
% Renseigner si le Rapport contient une partie 4
\setboolean{pt4}{true}
% Renseigner si le Rapport contient une partie 5
\setboolean{pt5}{true}
% Renseigner si le Rapport contient une introduction
\setboolean{conclusion}{true}
% Renseigner si le Rapport contient des perspectives
\setboolean{perspectives}{true}
% Renseigner si le document contient une bibliographie
\setboolean{biblio}{true} 
% Renseigner si le document contient un glossaire
\setboolean{glossaire}{false}
% Renseigner si le Rapport contient des annexes 
\setboolean{annexe}{true}


%%%%%%%%%%%%%%%%%%%%%%%%%%%%%%%%%%%%%%
%     En-têtes en pieds de pages     %
%%%%%%%%%%%%%%%%%%%%%%%%%%%%%%%%%%%%%%
\geometry{hmargin=2cm,vmargin=2.3cm}
\pagestyle{fancy}
\fancyhfoffset[]{0pt}
\setlength{\headheight}{28pt}
\lhead{\includegraphics[height = 0.6cm]{IMAGES/logos/Logo_SeaTech_2023.png}}
% \rhead{\includegraphics[height = 0.7cm]{IMAGES/logos/MOCA.png}}
\rhead{\textsc{\leftmark}}

% Update \rightmark with \section name
\renewcommand{\sectionmark}[1]{\markboth{#1}{#1}}


\lfoot{\auteur}
\cfoot{ }
\rfoot{Page \thepage \ / \pageref{LastPage}}

\title{\titre}
\author{\auteur}
\date{\today}

%%%%%%%%%%%%%%%%%%%%%%%%%%%%%%
%     Autre mise en page     %
%%%%%%%%%%%%%%%%%%%%%%%%%%%%%%
\numberwithin{figure}{section}
\numberwithin{table}{section}

\setcounter{tocdepth}{2} % Change to 1 to exclude subsections as well


\newcommand{\citeURL}[1]{\href{#1}{\detokenize{#1}}}

% Création du compteur d'annexes
\newcounter{annexecounter}

% Définition de la commande pour les annexes
\NewDocumentCommand{\annexe}{m}{%
    \stepcounter{annexecounter} % Incrémenter le compteur d'annexes
    \subsection*{Annexe \arabic{annexecounter} - #1} % Affichage du texte avec le numéro et le titre
	\label{sec:#1}
}

\newcommand{\tobedone}{\textcolor{red}{\LARGE \textbf{TO BE DONE}}}
\newcommand{\annexetonum}{\textcolor{red}{\LARGE \textbf{ANNEXE ...}}}
\newcommand{\figtonum}{\textcolor{red}{\textbf{FIGURE ...}}}

\renewcommand{\familydefault}{\sfdefault}



%%%%%%%%%%%%%%%%%%%%%%%%%%%%%%%%%
%     Mise en page des codes    %
%%%%%%%%%%%%%%%%%%%%%%%%%%%%%%%%%
\input{ANNEXES/codes/display/python_code.tex}
\input{ANNEXES/codes/display/cpp_code.tex}
\input{ANNEXES/codes/display/f90_code.tex}


%%%%%%%%%%%%%%%%%%%%%%%%%%%%%%%%%%%%%%%%%%%%%%%%%%%%%%%%%%%%%%%%%%%%%%%%%%%%%%%%%%%%%%%%%%%%%%%%%%%%%%%%%%%%%%%%%%%%%%%
%                                                  Début du document                                                  %
%%%%%%%%%%%%%%%%%%%%%%%%%%%%%%%%%%%%%%%%%%%%%%%%%%%%%%%%%%%%%%%%%%%%%%%%%%%%%%%%%%%%%%%%%%%%%%%%%%%%%%%%%%%%%%%%%%%%%%%

\begin{document}

%%%%%%%%%%%%%%%%%%%%%%%%%
%     Page de garde     %
%%%%%%%%%%%%%%%%%%%%%%%%%
\begin{titlepage}
	\AddToShipoutPictureBG*{\includegraphics[width=\paperwidth,height=\paperheight]{IMAGES/PageDeGardeRapport.png}}
	\begin{figure}[H]
		\begin{subfigure}{0.45\linewidth}
				\includegraphics[width=0.6\textwidth,left]{IMAGES/logos/Logo_SeaTech_2023.png}
		\end{subfigure}
		\hfill
		\begin{subfigure}{0.45\linewidth}
				% \includegraphics[width=0.6\textwidth,right]{IMAGES/logos/MOCA.png}
		\end{subfigure}
	\end{figure}

	\centering

	% Espacement vertical
	\vspace*{5cm}

	% Barres horizontales
	\makebox[0.7\linewidth]{\hrulefill}\\[0.2cm]

	% Titre encadré
	\vspace{0.5cm}
	\begin{minipage}{\textwidth}
		\centering
		{\fontsize{28}{48}\selectfont \textsc{\titre}}\\[0.2cm]

		{\fontsize{18}{48}\selectfont \textsc{\soustitre}}
	\end{minipage}
	\vspace{0.3cm}

	% Barres horizontales
	\makebox[0.8\linewidth]{\hrulefill}\\[0.2cm]

	% Espacement vertical
	\vspace{3cm}

	% Description
		\large{\Large \textbf{\sujet}}\\
		\large{\textbf{\sujets}}\\

		\vspace{0.5cm}
		\large{\textbf{\reportdate}}

	\vspace{2cm}

	\begin{minipage}{0.20\textwidth}

	\end{minipage}
	\hfill
	\begin{minipage}{0.35\textwidth}
		\begin{flushleft}
			Auteur : \\
			\auteur
		\end{flushleft}
	\end{minipage}
	\begin{minipage}{0.09\textwidth}
		% Section vide pour espacement optimal
	\end{minipage}
	\hfill
 	\begin{minipage}{0.3\textwidth}
		\begin{flushleft}
			Enseignant : \\
			\referent

		\end{flushleft}
	\end{minipage}


\end{titlepage}

\ClearShipoutPictureBG

\newpage

\renewcommand{\thepage}{}

\renewcommand{\thepage}{\arabic{page}}
\renewcommand{\thesection}{\Roman{section}}

%%%%%%%%%%%%%%%%%%
%     Résumé     %
%%%%%%%%%%%%%%%%%%
\ifthenelse{\boolean{abst}}{
	\addcontentsline{toc}{section}{\protect\numberline{}Résumé}%
	\subfile{SECTIONS/1resume}

	\newpage
}

%%%%%%%%%%%%%%%%%%%%%%%%%
%     Remerciements     %
%%%%%%%%%%%%%%%%%%%%%%%%%
\ifthenelse{\boolean{thx}}{
	\addcontentsline{toc}{section}{\protect\numberline{}Remerciements}%
	\subfile{SECTIONS/2remerciements}

	\newpage
}

%%%%%%%%%%%%%%%%%%%%%%%%%%%%
%     Plan du document     %
%%%%%%%%%%%%%%%%%%%%%%%%%%%%

\ifthenelse{\boolean{contents}}{
	\vfill
	\tableofcontents
	\vfill
	
	\newpage
}

%%%%%%%%%%%%%%%%%%%%%%%%
%     INTRODUCTION     %
%%%%%%%%%%%%%%%%%%%%%%%%
\ifthenelse{\boolean{introduction}}
{
	\addcontentsline{toc}{section}{\protect\numberline{}Introduction}%
	\section*{Introduction}

	\markboth{Introduction}{Introduction} % Manually update \rightmark for section*
	\subfile{SECTIONS/3introduction}


	\newpage
}


%%% PARTIE 1 %%%
\section{\partA}
\subfile{SECTIONS/part1}

%%% PARTIE 2 %%%
\newpage
\ifthenelse{\boolean{pt2}}
{
	\section{\partB}
	\subfile{SECTIONS/part2}
	
	\newpage
}
	
	
%%% PARTIE 3 %%%
\ifthenelse{\boolean{pt3}}
{
	\section{\partC}
	\subfile{SECTIONS/part3}
	
	\newpage
}
	
	
%%% PARTIE 4 %%%
\ifthenelse{\boolean{pt4}}
{
	\section{\partD}
	\subfile{SECTIONS/part4}
	
	\newpage
}
	
%%% PARTIE 5 %%%
\ifthenelse{\boolean{pt5}}{
	\section{\partE}
	\subfile{SECTIONS/part5}

	\newpage
}
		
%%%%%%%%%%%%%%%%%%%%%%
%     CONCLUSION     %
%%%%%%%%%%%%%%%%%%%%%%
\ifthenelse{\boolean{conclusion}}
{
	\addcontentsline{toc}{section}{\protect\numberline{}Conclusion}%
	\section*{Conclusion}
	\markboth{Conclusion}{Conclusion} % Manually update \rightmark for section*
	
	\subfile{SECTIONS/Wconclusion}

	\newpage
}



\ifthenelse{\boolean{perspectives}}
{
	\section*{Perspectives}
	\markboth{Perspectives}{Perspectives} % Manually update \rightmark for section*
	\addcontentsline{toc}{section}{\protect\numberline{}Perspectives}
	\subfile{SECTIONS/Xperspectives}
	
	\newpage 
}

%%%%%%%%%%%%%%%%%%%%%%%%%
%     Bibliographie     %
%%%%%%%%%%%%%%%%%%%%%%%%%

\ifthenelse{\boolean{biblio}}
{
	\addcontentsline{toc}{section}{\protect\numberline{}References}
	% \bibliographystyle{unsrt}
	\bibliographystyle{IEEEtran}
	\footnotesize{\bibliography{BIBLIOGRAPHY/bib.bib}}

	\newpage
}


%%%%%%%%%%%%%%%%%%%%%
%     Glossaire     %
%%%%%%%%%%%%%%%%%%%%%
\normalsize
\ifthenelse{\boolean{glossaire}}
{
	\section*{Glossaire}
	\makeglossaries
	\printglossaries
	\addcontentsline{toc}{section}{\protect\numberline{}Glossaire}%
	\subfile{SECTIONS/Yglossaire}
	
	\newpage
}

%%%%%%%%%%%%%%%%%%%
%     Annexes     %
%%%%%%%%%%%%%%%%%%%
\ifthenelse{\boolean{annexe}}
{
	\section*{Annexes}
	\addcontentsline{toc}{section}{\protect\numberline{}Annexes}%
	\subfile{SECTIONS/Zannexes}
}


\end{document}


\begin{document}

\subsection{État de l'Art : Exploration Multi-Robot de Cavités}

L'exploration des cavités est essentielle pour la cartographie souterraine, la recherche scientifique et les interventions d'urgence, tout en posant des défis uniques pour la robotique autonome.\cite{nationalgeographic_greenland_caves_2025}
Leur exploration multi-robot est un domaine de recherche en plein essor dans le domaine de la robotique, avec des applications dans des environnements variés tels que l'exploration spatiale, l'exploration de grottes, les missions de sauvetage dans des environnements urbains sinistrés, ou encore l'extraction minière.\cite{dang_2021,kambesis_2007} Ces environnements, souvent complexes et dynamiques, présentent de nombreux défis pour la planification de mission, notamment des obstacles imprévisibles, des zones inaccessibles pour les robots mobiles, et l'absence de signal GPS. De ce fait, l'exploration multi-robot permet de tirer parti de la coopération entre robots pour surmonter ces difficultés.

\subsubsection{Défis dans l'exploration multi-robot de cavités}

L'exploration de cavités avec plusieurs robots implique plusieurs défis techniques. En particulier, la gestion de la coopération entre robots et la gestion de l'information dans des environnements complexes sont deux aspects fondamentaux de ce type de mission. Les défis peuvent être classés en plusieurs catégories : 

\subsection{État de l'Art des Techniques de SLAM}

Le Simultaneous Localization and Mapping (SLAM) est un problème clé en robotique qui permet à un robot de localiser sa position tout en construisant une carte de son environnement. Plusieurs techniques ont été développées pour résoudre ce problème, telles que le EKF-SLAM, qui utilise un filtre de Kalman pour estimer la localisation et la carte, mais souffre de limitations en termes de scalabilité dans de grands environnements. Le FastSLAM, basé sur un filtre de particules, améliore cette scalabilité en traitant les caractéristiques du monde à l'aide de multiples hypothèses. Les approches basées sur les graphes, comme le Graph-SLAM, sont efficaces pour résoudre des problèmes d'optimisation à grande échelle, bien qu'elles nécessitent une gestion complexe des données. Le SLAM visuel (V-SLAM) repose sur des caméras pour estimer la localisation et construire une carte, tandis que le SLAM LiDAR utilise des capteurs laser pour des mesures de profondeur précises, particulièrement efficaces pour des environnements extérieurs. Des variantes comme le SLAM dynamique gèrent les environnements avec des objets mobiles en excluant ces derniers des mises à jour de la carte. Ces méthodes varient en fonction des environnements, des ressources du robot et des contraintes en temps réel, avec pour objectif de maintenir une localisation et une cartographie précises dans des conditions dynamiques et complexes.


\subsection{État de l'Art des Méthodes de Communication Inter-Robot dans un Environnement de Cavité}

La communication inter-robot (CIR) dans un environnement de cavité est un défi majeur pour les systèmes multi-robots, en raison des conditions particulières que ces environnements présentent, telles que des espaces confinés, des obstacles physiques et des perturbations qui peuvent affecter les signaux de communication. Les méthodes de CIR peuvent être classées en différentes catégories selon la technologie utilisée et la stratégie de communication adoptée. Les approches classiques reposent principalement sur des réseaux sans fil, tels que la communication par radiofréquence (RF), qui est couramment utilisée pour des applications de communication à longue portée mais qui peut souffrir de limitations dans les cavités où les signaux sont atténués par des parois solides. Dans ce contexte, des méthodes de communication ad hoc sont souvent employées, où les robots créent un réseau dynamique de relais pour échanger des informations. Une approche courante est l'utilisation de la \textit{communication par maillage}, dans laquelle chaque robot agit comme un relais, permettant ainsi une couverture étendue et une transmission de données entre robots même lorsque les obstacles interfèrent avec les signaux directs. La \textit{communication acoustique} ou \textit{ultrasonique}, qui repose sur des ondes sonores, est une alternative viable dans les environnements de cavité, où elle peut transmettre des informations de manière robuste, notamment pour des distances plus courtes ou dans des environnements très confinés. De plus, la communication optique (par exemple, la \textit{communication par lumière visible ou infrarouge}) devient de plus en plus populaire dans des applications de haute précision dans des cavités étroites, car elle est peu sensible aux interférences électromagnétiques et permet des transmissions rapides et sécurisées. Une autre stratégie consiste à utiliser des protocoles de communication coopérative, où les robots collaborent pour optimiser le flux d'informations. Des techniques de \textit{routage adaptatif} et de \textit{partitionnement dynamique de réseau} sont souvent utilisées pour gérer les obstacles et les interférences dans les réseaux multi-robots. Enfin, les méthodes de \textit{communication opportuniste}, basées sur des échanges de données ponctuels lorsque la ligne de visée est dégagée, sont particulièrement adaptées aux environnements où la connectivité est intermittente et où les robots doivent s'organiser pour minimiser les interruptions dans le flux de données. L'un des principaux défis reste de maintenir une communication fiable et efficace, notamment en optimisant les stratégies de répartition de la bande passante, d'allocation des ressources et de gestion des interférences dans des espaces très contraints. Des approches récentes cherchent à intégrer des systèmes hybrides combinant ces technologies pour assurer une meilleure résilience face aux conditions environnementales changeantes et aux mouvements des robots. 


\subsubsection{Planification de chemin, de trajectoire et évitement d'obstacle}

La planification de chemin pour un système méchatronique constitue le fondement de tous les systèmes mobiles autonomes, qu'il s'agisse de drones ou de bras de manutention et d'assemblage. Le principe de la planification de chemin ou de trajectoire est de déterminer une solution - un chemin ou une trajectoire - reliant un point de départ à un point cible.

La distinction entre planification de chemin et planification de trajectoire réside dans le fait qu'un chemin planifié n'est pas nécessairement réalisable par un robot. En effet, la planification de chemin ne prend pas toujours en compte la faisabilité physique ou cinématique pour un robot mobile.

L'évitement d'obstacles est une contrainte essentielle dans ces deux approches. Les obstacles définissent les zones inaccessibles, et comme nous le verrons par la suite, leur nature - mobile ou immobile - détermine en grande partie la méthode à employer pour résoudre le problème de planification.\\

Parmis les méthodes déterministe, on trouve une large variete de méthodes principalement basées sur trois approches différentes. Les approches par graphs, celle de décomposition célullaire et enfin les celles utilisant des champs potentiels.\cite{latombe_robot_1991,bhattacharyya_robot_2008}

La méthodes des graphs consistent à construire une carte des chemins empruntable en partant des obstacles de la scène. Parmis ces méthodes utilisant des graphs, on peut distinguer quatres types differents : Les graph de visibilité\cite{visibility_graph_1979}, les diagrammes de Voronoï \cite{garrido_path_2006} ou encore la méthode des Silhouette\cite{bhattacharyya_robot_2008}.

Les méthodes associés au decomposition celulaire consistent à diviser l'espace libre du robot en régions simples, appelées cellules, où il est facile de générer un chemin entre deux configurations. Un graphe représentant les relations d'adjacence entre les cellules est ensuite construit et exploré.\cite{latombe_robot_1991,zhu_new_1991,kedem_efficient_1990,avnaim_practical_1988}

Une autre méthode repose sur une subdivision fine de l'espace afin de repérer les zones libres. La méthode des champs potentiels s'appuie sur cette idée en définissant des potentiels qui traduisent des forces d'attraction, dirigées vers les coordonnées cibles, et de répulsion, correspondant par exemple aux obstacles. Le chemin est ensuite déterminé en suivant l'opposé du gradient du potentiel total ainsi calculé.\cite{latombe_robot_1991,koren_potential_1991}

Des approches alternatives ont été développées dans les années 1990 et 2000, notamment les méthodes stochastiques de planification de chemin RPP (\textit{Random Path Planners}) et PRM (\textit{Probabilistic Roadmap Planners}). Ces méthodes consistent à échantillonner l'espace de manière aléatoire afin de créer un graphe de chemins possibles (\textit{roadmap}) dans cet espace.\cite{amato_1996,hsu_2002,nissoux_1999}  

L'un des principaux atouts de cette méthode de planification de chemin est qu'elle ne dépend ni de la structure de l'espace, ni du nombre d'obstacles, ni de leur disposition. Une fois la \textit{roadmap} créée, il suffit d'utiliser une méthode de recherche de chemin dans un graphe pour trouver le chemin menant au point objectif.\cite{gasparetto_2015}


\end{document}