\LoadClass[a4paper,12pt]{article}
\documentclass{article}
%%%%%%%%%%%%%%%%%%%%%%%%%%%%%%%
%     import des packages     %
%%%%%%%%%%%%%%%%%%%%%%%%%%%%%%%
\usepackage[export]{adjustbox}
\usepackage{algorithm}
\usepackage{algorithmic}
\usepackage{amsmath,amsfonts,amssymb}
\usepackage{anyfontsize}
\usepackage{array}
\usepackage[english]{babel}
\usepackage{colortbl}
\usepackage{comment}
\usepackage{cclicenses}
\usepackage{eqnarray}
\usepackage{eso-pic}
\usepackage{dirtree}
\usepackage{fancybox}
\usepackage{fancyhdr}
\usepackage{float}
\usepackage[T1]{fontenc} 
\usepackage{forest}
\usepackage{fourier-orns}
\usepackage{gensymb}
\usepackage{geometry}
\usepackage{glossaries}
\usepackage{graphicx}
\usepackage{hyperref}
\usepackage{ifthen}
\usepackage{import}
\usepackage{indentfirst}
\usepackage[utf8]{inputenc}
\usepackage{lastpage}
\usepackage{libertine}
\usepackage{lipsum}
\usepackage{listings}
\usepackage{mathtools}
\usepackage{mdframed}
\usepackage{multicol}
\usepackage{pdfpages}
\usepackage{pifont}
\usepackage{stmaryrd}
\usepackage{subcaption}
\usepackage{subfiles}
\usepackage{tabularx}
% \usepackage{tcolorbox}
\usepackage[most]{tcolorbox}
\usepackage{textcomp}
\usepackage{tikz}
\usepackage{titlesec}
\usepackage{ulem}
\usepackage{wrapfig}


\numberwithin{figure}{section}
\numberwithin{table}{section}


%%%%%%%%%%%%%%%%%%%%%%%%%%%%%%%%%%%%%%
%     En-têtes en pieds de pages     %
%%%%%%%%%%%%%%%%%%%%%%%%%%%%%%%%%%%%%%
\geometry{hmargin=2cm,vmargin=2.3cm}
\pagestyle{fancy}
\fancyhfoffset[]{0pt}
\setlength{\headheight}{28pt}
\lfoot{Fabien MATHÉ}
\cfoot{ }
\rfoot{Page \thepage \ / \pageref{LastPage}}


\renewcommand{\thesection}{\Roman{section}}
\renewcommand{\familydefault}{\sfdefault}

%%%%%%%%%%%%%%%%%%%%%%%%%%%%%t
%     Début du document     %
%%%%%%%%%%%%%%%%%%%%%%%%%%%%%

\begin{document}

\begin{titlepage}
    \begin{center}
        \Large
        \vfill
        \textsc{Conception et optimisation d'algorithmes de coordination multi-robot pour l'exploration autonome de réseaux de galerie.}
        \vfill
    \end{center}
\end{titlepage}
\newpage

\subfile{state_of_the_art}

\newpage

\subfile{math_def}

\newpage

\subfile{avancement}

\newpage

\section{Partie à développer}

\subsection*{Création du monde}
\begin{itemize}
    \item Automate cellulaire
    \item Méthode de square marching appliquée à des triangles
\end{itemize}

\subsection*{Capteurs utilisés}
Minimalisme, quels capteurs sont nécessaire et suffisant pour accomplir la mission donnée ?
\begin{itemize}
    \item Fonctionnement d'un lidar
    \item Accéléromètre
\end{itemize}

\subsection*{Planification de trajectoire}
\begin{itemize}
	\item \textbf{Différence avec la plannification de chemin (si existe ?)}
    \item Définition de la trajectoire optimale
    \item Description mathématique dans notre cas
    \item Manière de le résoudre, temps nécessaire, possibilités
\end{itemize}

\subsection*{Évitement d'obstacles}
\subsubsection*{Essais}
\begin{itemize}
    \item Diagramme de Voronoi
    \item DFS (Depth-First Search)    
\end{itemize}

\subsubsection*{Méthode par subdivision successive}

\begin{itemize}
	\item Isotropic waves generator to validate or reject the method
		\item Cellular automata
\end{itemize}

\subsubsection*{Création d'une carte de caractéristique sur un maillage cartésien uniforme}
\subsubsection*{Planification dynamique de trajectoire}

\subsection*{Contrôle du robot}
\begin{itemize}
    \item Cinématique du robot utilisé
    \item Modèle intégré
\end{itemize}

\subsection*{Temps de calcul}
\begin{itemize}
    \item Liste des optimisations
    \begin{itemize}
        \item Subdivision de la carte
        \item Détection d'obstacles par le Lidar
        \begin{itemize}
            \item Fonction \texttt{move\_on\_line} (parallèle avec le \emph{computer graphics})
        \end{itemize}
    \end{itemize}
\end{itemize}


\subsubsection*{Non encore implémenté}
\begin{itemize}
    \item Sauvegarde dynamique de l'exploration
\end{itemize}




\newpage
\addcontentsline{toc}{section}{\protect\numberline{}References}
\bibliographystyle{unsrt}
\footnotesize{\bibliography{bib.bib}}


\end{document}