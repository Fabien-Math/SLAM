\documentclass[main.tex]{subfiles}
\LoadClass[a4paper,12pt]{article}
\documentclass{article}

\usepackage[english]{babel}

%%%%%%%%%%%%%%%%%%%%%%%%%%%%%%%
%     import des packages     %
%%%%%%%%%%%%%%%%%%%%%%%%%%%%%%%
\usepackage[export]{adjustbox}
\usepackage{algorithm}
\usepackage{algorithmic}
\usepackage{amsmath,amsfonts,amssymb}
\usepackage{anyfontsize}
\usepackage{array}
\usepackage[french]{babel}
\usepackage{colortbl}
\usepackage{comment}
\usepackage{cclicenses}
\usepackage{eqnarray}
\usepackage{eso-pic}
\usepackage{dirtree}
\usepackage{fancybox}
\usepackage{fancyhdr}
\usepackage{float}
\usepackage[T1]{fontenc} 
\usepackage{forest}
\usepackage{fourier-orns}
\usepackage{gensymb}
\usepackage{geometry}
\usepackage{glossaries}
\usepackage{graphicx}
\usepackage{hyperref}
\usepackage{ifthen}
\usepackage{import}
\usepackage{indentfirst}
\usepackage[utf8]{inputenc}
\usepackage{lastpage}
\usepackage{libertine}
\usepackage{lipsum}
\usepackage{listings}
\usepackage{mathtools}
\usepackage{mdframed}
\usepackage{multicol}
\usepackage{pdfpages}
\usepackage{pifont}
\usepackage{stmaryrd}
\usepackage{subcaption}
\usepackage{subfiles}
\usepackage{tabularx}
% \usepackage{tcolorbox}
\usepackage[most]{tcolorbox}
\usepackage{textcomp}
\usepackage{ulem}
\usepackage{wrapfig}

%%%%%%%%%%%%%%%%%%%%%%%%%%%%%%%%%%%%%%%%%%%%%%%%%%%%%%%%%
%    Renseigner les titres et variables importantes     %
%%%%%%%%%%%%%%%%%%%%%%%%%%%%%%%%%%%%%%%%%%%%%%%%%%%%%%%%%
\newcommand{\titre}{Multi-robot coordination}
\newcommand{\soustitre}{Autonomous exploration of gallery networks}
\newcommand{\sujet}{Engineering Graduation Project}
\newcommand{\sujets}{Seatech 3A - MOCA}
\newcommand{\auteur}{Fabien MATHÉ}
\newcommand{\referent}{M. Mehmet ERSOY}
\newcommand{\reportdate}{\date}

\newcommand{\partA}{State of the art}
\newcommand{\partB}{Notations}
\newcommand{\partC}{Path planning}
\newcommand{\partD}{Communication}
\newcommand{\partE}{Simulator implementation}

%%%%%%%%%%%%%%%%%%%
%     BOOLEEN     %
%%%%%%%%%%%%%%%%%%%

% Renseigner si le Rapport contient un abstract
\setboolean{abst}{true}
% Renseigner si le Rapport contient des remerciements
\setboolean{thx}{true}
% Renseigner si le Rapport contient une table des matières
\setboolean{contents}{true}
% Renseigner si le Rapport contient une introduction
\setboolean{introduction}{true}
% Renseigner si le Rapport contient une partie 2
\setboolean{pt2}{true}
% Renseigner si le Rapport contient une partie 3
\setboolean{pt3}{true}
% Renseigner si le Rapport contient une partie 4
\setboolean{pt4}{true}
% Renseigner si le Rapport contient une partie 5
\setboolean{pt5}{true}
% Renseigner si le Rapport contient une introduction
\setboolean{conclusion}{true}
% Renseigner si le Rapport contient des perspectives
\setboolean{perspectives}{true}
% Renseigner si le document contient une bibliographie
\setboolean{biblio}{true} 
% Renseigner si le document contient un glossaire
\setboolean{glossaire}{false}
% Renseigner si le Rapport contient des annexes 
\setboolean{annexe}{true}


%%%%%%%%%%%%%%%%%%%%%%%%%%%%%%%%%%%%%%
%     En-têtes en pieds de pages     %
%%%%%%%%%%%%%%%%%%%%%%%%%%%%%%%%%%%%%%
\geometry{hmargin=2cm,vmargin=2.3cm}
\pagestyle{fancy}
\fancyhfoffset[]{0pt}
\setlength{\headheight}{28pt}
\lhead{\includegraphics[height = 0.6cm]{IMAGES/logos/Logo_SeaTech_2023.png}}
% \rhead{\includegraphics[height = 0.7cm]{IMAGES/logos/MOCA.png}}
\rhead{\textsc{\leftmark}}

% Update \rightmark with \section name
\renewcommand{\sectionmark}[1]{\markboth{#1}{#1}}


\lfoot{\auteur}
\cfoot{ }
\rfoot{Page \thepage \ / \pageref{LastPage}}

\title{\titre}
\author{\auteur}
\date{\today}

%%%%%%%%%%%%%%%%%%%%%%%%%%%%%%
%     Autre mise en page     %
%%%%%%%%%%%%%%%%%%%%%%%%%%%%%%
\numberwithin{figure}{section}
\numberwithin{table}{section}

\setcounter{tocdepth}{2} % Change to 1 to exclude subsections as well


\newcommand{\citeURL}[1]{\href{#1}{\detokenize{#1}}}

% Création du compteur d'annexes
\newcounter{annexecounter}

% Définition de la commande pour les annexes
\NewDocumentCommand{\annexe}{m}{%
    \stepcounter{annexecounter} % Incrémenter le compteur d'annexes
    \subsection*{Annexe \arabic{annexecounter} - #1} % Affichage du texte avec le numéro et le titre
	\label{sec:#1}
}

\newcommand{\tobedone}{\textcolor{red}{\LARGE \textbf{TO BE DONE}}}
\newcommand{\annexetonum}{\textcolor{red}{\LARGE \textbf{ANNEXE ...}}}
\newcommand{\figtonum}{\textcolor{red}{\textbf{FIGURE ...}}}

\renewcommand{\familydefault}{\sfdefault}



%%%%%%%%%%%%%%%%%%%%%%%%%%%%%%%%%
%     Mise en page des codes    %
%%%%%%%%%%%%%%%%%%%%%%%%%%%%%%%%%
\input{ANNEXES/codes/display/python_code.tex}
\input{ANNEXES/codes/display/cpp_code.tex}
\input{ANNEXES/codes/display/f90_code.tex}


%%%%%%%%%%%%%%%%%%%%%%%%%%%%%%%%%%%%%%%%%%%%%%%%%%%%%%%%%%%%%%%%%%%%%%%%%%%%%%%%%%%%%%%%%%%%%%%%%%%%%%%%%%%%%%%%%%%%%%%
%                                                  Début du document                                                  %
%%%%%%%%%%%%%%%%%%%%%%%%%%%%%%%%%%%%%%%%%%%%%%%%%%%%%%%%%%%%%%%%%%%%%%%%%%%%%%%%%%%%%%%%%%%%%%%%%%%%%%%%%%%%%%%%%%%%%%%

\begin{document}

%%%%%%%%%%%%%%%%%%%%%%%%%
%     Page de garde     %
%%%%%%%%%%%%%%%%%%%%%%%%%
\begin{titlepage}
	\AddToShipoutPictureBG*{\includegraphics[width=\paperwidth,height=\paperheight]{IMAGES/PageDeGardeRapport.png}}
	\begin{figure}[H]
		\begin{subfigure}{0.45\linewidth}
				\includegraphics[width=0.6\textwidth,left]{IMAGES/logos/Logo_SeaTech_2023.png}
		\end{subfigure}
		\hfill
		\begin{subfigure}{0.45\linewidth}
				% \includegraphics[width=0.6\textwidth,right]{IMAGES/logos/MOCA.png}
		\end{subfigure}
	\end{figure}

	\centering

	% Espacement vertical
	\vspace*{5cm}

	% Barres horizontales
	\makebox[0.7\linewidth]{\hrulefill}\\[0.2cm]

	% Titre encadré
	\vspace{0.5cm}
	\begin{minipage}{\textwidth}
		\centering
		{\fontsize{28}{48}\selectfont \textsc{\titre}}\\[0.2cm]

		{\fontsize{18}{48}\selectfont \textsc{\soustitre}}
	\end{minipage}
	\vspace{0.3cm}

	% Barres horizontales
	\makebox[0.8\linewidth]{\hrulefill}\\[0.2cm]

	% Espacement vertical
	\vspace{3cm}

	% Description
		\large{\Large \textbf{\sujet}}\\
		\large{\textbf{\sujets}}\\

		\vspace{0.5cm}
		\large{\textbf{\reportdate}}

	\vspace{2cm}

	\begin{minipage}{0.20\textwidth}

	\end{minipage}
	\hfill
	\begin{minipage}{0.35\textwidth}
		\begin{flushleft}
			Auteur : \\
			\auteur
		\end{flushleft}
	\end{minipage}
	\begin{minipage}{0.09\textwidth}
		% Section vide pour espacement optimal
	\end{minipage}
	\hfill
 	\begin{minipage}{0.3\textwidth}
		\begin{flushleft}
			Enseignant : \\
			\referent

		\end{flushleft}
	\end{minipage}


\end{titlepage}

\ClearShipoutPictureBG

\newpage

\renewcommand{\thepage}{}

\renewcommand{\thepage}{\arabic{page}}
\renewcommand{\thesection}{\Roman{section}}

%%%%%%%%%%%%%%%%%%
%     Résumé     %
%%%%%%%%%%%%%%%%%%
\ifthenelse{\boolean{abst}}{
	\addcontentsline{toc}{section}{\protect\numberline{}Résumé}%
	\subfile{SECTIONS/1resume}

	\newpage
}

%%%%%%%%%%%%%%%%%%%%%%%%%
%     Remerciements     %
%%%%%%%%%%%%%%%%%%%%%%%%%
\ifthenelse{\boolean{thx}}{
	\addcontentsline{toc}{section}{\protect\numberline{}Remerciements}%
	\subfile{SECTIONS/2remerciements}

	\newpage
}

%%%%%%%%%%%%%%%%%%%%%%%%%%%%
%     Plan du document     %
%%%%%%%%%%%%%%%%%%%%%%%%%%%%

\ifthenelse{\boolean{contents}}{
	\vfill
	\tableofcontents
	\vfill
	
	\newpage
}

%%%%%%%%%%%%%%%%%%%%%%%%
%     INTRODUCTION     %
%%%%%%%%%%%%%%%%%%%%%%%%
\ifthenelse{\boolean{introduction}}
{
	\addcontentsline{toc}{section}{\protect\numberline{}Introduction}%
	\section*{Introduction}

	\markboth{Introduction}{Introduction} % Manually update \rightmark for section*
	\subfile{SECTIONS/3introduction}


	\newpage
}


%%% PARTIE 1 %%%
\section{\partA}
\subfile{SECTIONS/part1}

%%% PARTIE 2 %%%
\newpage
\ifthenelse{\boolean{pt2}}
{
	\section{\partB}
	\subfile{SECTIONS/part2}
	
	\newpage
}
	
	
%%% PARTIE 3 %%%
\ifthenelse{\boolean{pt3}}
{
	\section{\partC}
	\subfile{SECTIONS/part3}
	
	\newpage
}
	
	
%%% PARTIE 4 %%%
\ifthenelse{\boolean{pt4}}
{
	\section{\partD}
	\subfile{SECTIONS/part4}
	
	\newpage
}
	
%%% PARTIE 5 %%%
\ifthenelse{\boolean{pt5}}{
	\section{\partE}
	\subfile{SECTIONS/part5}

	\newpage
}
		
%%%%%%%%%%%%%%%%%%%%%%
%     CONCLUSION     %
%%%%%%%%%%%%%%%%%%%%%%
\ifthenelse{\boolean{conclusion}}
{
	\addcontentsline{toc}{section}{\protect\numberline{}Conclusion}%
	\section*{Conclusion}
	\markboth{Conclusion}{Conclusion} % Manually update \rightmark for section*
	
	\subfile{SECTIONS/Wconclusion}

	\newpage
}



\ifthenelse{\boolean{perspectives}}
{
	\section*{Perspectives}
	\markboth{Perspectives}{Perspectives} % Manually update \rightmark for section*
	\addcontentsline{toc}{section}{\protect\numberline{}Perspectives}
	\subfile{SECTIONS/Xperspectives}
	
	\newpage 
}

%%%%%%%%%%%%%%%%%%%%%%%%%
%     Bibliographie     %
%%%%%%%%%%%%%%%%%%%%%%%%%

\ifthenelse{\boolean{biblio}}
{
	\addcontentsline{toc}{section}{\protect\numberline{}References}
	% \bibliographystyle{unsrt}
	\bibliographystyle{IEEEtran}
	\footnotesize{\bibliography{BIBLIOGRAPHY/bib.bib}}

	\newpage
}


%%%%%%%%%%%%%%%%%%%%%
%     Glossaire     %
%%%%%%%%%%%%%%%%%%%%%
\normalsize
\ifthenelse{\boolean{glossaire}}
{
	\section*{Glossaire}
	\makeglossaries
	\printglossaries
	\addcontentsline{toc}{section}{\protect\numberline{}Glossaire}%
	\subfile{SECTIONS/Yglossaire}
	
	\newpage
}

%%%%%%%%%%%%%%%%%%%
%     Annexes     %
%%%%%%%%%%%%%%%%%%%
\ifthenelse{\boolean{annexe}}
{
	\section*{Annexes}
	\addcontentsline{toc}{section}{\protect\numberline{}Annexes}%
	\subfile{SECTIONS/Zannexes}
}


\end{document}


\begin{document}

\section{Avancement}

\subsection{1ère semaine :}

\begin{itemize}
    \item Recherche documentaire autour des méthodes pour définir l'espace mathématique dans lequel le robot évolu
    \item Recherche documentaire autour des méthodes d'exploration multi-robot
    \item Définition de l'espace étoilé relatif champ de vision du robot
    \item Définition de la fonctionnelle du robot
    \item 
\end{itemize}


\subsection{2ème semaine :}

\begin{itemize}
    \item Début de la rédaction de l'état de l'art autour des méthode de planification de trajectoire
    \item Définition de la fonctionnelle $J$
    \item Définition de la borne supérieur de l'intégrale de $J$
    \item Correction du comportement du lidar pour l'exploration de la map
    \item Implentation des colision du robot avec la map
    \item 
\end{itemize}


Note sur les LiDAR (\textit{Wikipedia}): 

La télédétection par laser ou LIDAR, acronyme de l'expression en langue anglaise « light detection and ranging » ou « laser imaging detection and ranging » (soit en français « détection et estimation de la distance par la lumière » ou « par laser »), est une technique de mesure à distance fondée sur l'analyse des propriétés d'un faisceau de lumière renvoyé vers son émetteur.

À la différence du sonar qui utilise des ondes acoustiques et du radar qui emploie des ondes électromagnétiques de plus basse fréquence (ondes radio), le lidar utilise des ondes électromagnétiques proches de la lumière visible (du spectre visible, infrarouge ou ultraviolet). La lumière utilisée par le lidar est presque toujours issue d'un laser, ce qui permet d'avoir une source lumineuse directionnelle, monochromatique, polarisée, de haute amplitude et cohérente.

Le principe de la télémétrie (détermination de la distance d'un objet), qui concerne une grande partie des applications du lidar, requiert généralement l'utilisation d'un laser impulsionnel. La distance est donnée par la mesure du délai (aussi appelé « temps de vol ») entre l'émission d'une impulsion et la détection d'une impulsion réfléchie, connaissant la vitesse de la lumière. Quand la source laser est entretenue et modulée en fréquence, on parle de lidar FMCW (Frequency Modulated Continuous Wave). 


\subsubsection{Recherche à faire}

Faire un histogramme des temps de recouvrement de la map en definissant plusieurs critères ou de manière aléatoire afin de dégager des comportements qui semble "optimaux", recommencer afin de trouver le minimum global ou du moins local de manière expérimentale.

Implémentation de l'évitement d'obstacle.

Approfondir la compréhension autour de FMM et des équations Eikonale

Regarder comment le relier avec un portrait de phase, équation différentielle.
Tracer la trajectoire x1 en fonction de la trajectoire x2.
Portrait de phase de système dynamique aléatoire ? Est ce possible ?

Systeme dynamique avec controle (optimal dans la suite)


\section{Multi-robots control}

This algorithm aim to coordinate the exploration of unknown area by a robot swarm.

\subsection{Operation}

A the beginning of the exploration, we assign a number to each robot, the smallest one is the master. Another Implementation would be to choose the master accordingly to their battery level, the one with the higher battery level is the master. Indeed, the master robot is the one that consume more due to all the calculation it will do. To keep it simple, we choose the smallest number.

The communication is bases on a strong hierarchical structure, the master robot give the direction to all robot link with it with a bigger number.

If the group split, the master change to the strongest robot in the group.
Robot communicate with each other using light, detecting if the robot can communicate with another is made, in a first time, by ensuring visual contact. A more powerful approach is to simulate the travel of the light in the medium.

The master of a group is a hub for communication, each new direction is given by him using this method :

\begin{itemize}
	\item Each robot in a group sends the master any part of the map that is new to it
	\item Each robot of a group sends the master the waypoint with the open frontier index it wants to explore.
	\item Once all robot of the group send its information to the master, the protocol for gathering information is given later, it compute the normalized cost table for each combination robot-waypoint. Values are given between 0 and 255 to send only one byte information ensuring low communication volume. 
	
	\begin{table}[H]
		\centering
		\begin{tabular}{|c|c|c|c|c|c|}
			\hline
			& \textbf{Robot 1} & \textbf{Robot 2} & \textbf{Robot 3} & \textbf{Robot 4} & \textbf{Robot 5}\\
			\hline
			\textbf{WP 1} & \textcolor{blue}{$\mathbf{23}$} & $87$ & $234$ & $56$ & $192$ \\
			\hline
			\textbf{WP 2} & $245$ & $76$ & \textcolor{blue}{$\mathbf{11}$} & $68$ & $39$ \\
			\hline
			\textbf{WP 3} & $90$ & \textcolor{blue}{$\mathbf{21}$} & $73$ & $\mathbf{50}$ & $164$ \\
			\hline
			\textbf{WP 4} & $132$ & $58$ & $49$ & $77$ & \textcolor{blue}{$\mathbf{25}$} \\
			\hline
			\textbf{WP 5} & $181$ & $\mathbf{13}$ & $66$ & \textcolor{blue}{$\mathbf{39}$} & $70$ \\ 
			\hline
	
		\end{tabular}
		\caption{Example of costs table for a group of 5 robots with 5 differents waypoints}
		\label{tab:example_5R5WP}
	\end{table}
	
	In the case, there is less waypoint than robot, the group split in two groups.
	
	\begin{table}[H]
		\centering
		\begin{tabular}{|c|c|c|c|c|c|}
			\hline
			& \textbf{Robot 1} & \textbf{Robot 2} & \textbf{Robot 3} & \textbf{Robot 4} & \textbf{Robot 5}\\
			\hline
			\textbf{WP 1} & \textcolor{blue}{$\mathbf{61}$} & $125$ & $93$ & $47$ & \textcolor{blue}{$\mathbf{59}$} \\
			\hline
			\textbf{WP 2} & $88$ & \textcolor{blue}{$\mathbf{12}$} & \textcolor{blue}{$\mathbf{53}$} & \textcolor{blue}{$\mathbf{29}$} & $174$ \\ 
			\hline
		\end{tabular}
		\caption{Example of costs table for a group of 5 robots with 2 differents waypoints}
		\label{tab:example_5R2WP}
	\end{table}


	In the case, there is less robot than waypoint, each robot explore a zone.
	
	\begin{table}[H]
		\centering
		\begin{tabular}{|c|c|c|c|c|c|}
			\hline
			& \textbf{Robot 1} & \textbf{Robot 2} & \textbf{Robot 3} & \textbf{Robot 4} & \textbf{Robot 5}\\
			\hline
			\textbf{WP 1} & \textcolor{blue}{$\mathbf{42}$} & $134$ & $212$ & $63$ & $189$ \\
			\hline
			\textbf{WP 2} & $215$ & $98$ & $\mathbf{19}$ & $75$ & $48$ \\
			\hline
			\textbf{WP 3} & $102$ & \textcolor{blue}{$\mathbf{32}$} & $95$ & $68$ & $141$ \\
			\hline
			\textbf{WP 4} & $142$ & $74$ & $58$ & $89$ & $\mathbf{31}$ \\
			\hline
			\textbf{WP 5} & $193$ & $\mathbf{27}$ & $78$ & \textcolor{blue}{$\mathbf{54}$} & $83$ \\ 
			\hline
			\textbf{WP 6} & $156$ & $63$ & \textcolor{blue}{$\mathbf{17}$} & $99$ & $115$ \\
			\hline
			\textbf{WP 7} & $173$ & $49$ & $132$ & $82$ & \textcolor{blue}{$\mathbf{23}$} \\
			\hline
			\textbf{WP 8} & $204$ & $\mathbf{57}$ & $146$ & $71$ & $94$ \\ 
			\hline
		\end{tabular}
		\caption{Example of costs table for a group of 5 robots with 8 differents waypoints}
		\label{tab:example_5R8WP}
	\end{table}

	\item The master, after computing the cost table, distribute the waypoint across the group minimizing the cost combination. For instance, in the \autoref{tab:example_5R5WP}, \autoref{tab:example_5R2WP} and \autoref{tab:example_5R8WP}, bold number are the minimum cost for each robot and each waypoint, however the minimun combination of cost is given by the blue one.

\end{itemize}

\subsubsection{Meeting of two groups}

If 2 groups met, both master share all the information they have without moving. Once the trasfert is done, the strongest master take control of all the group and continue the exploration.




\subsection{Inter-robot communication protocol}
For each robot one at a time in the group.
The master asked for informations.
Listen for the answer.
If an answer is given, the master validate the tranfert of information.
Else, retry, retry and retry and skip it.
Compute thing.
Give an order to the robot.
The robot listen for the order.
If an answer is received, the robot validate the tranfert of information.
Else, retry, retry and retry and skip it.

\subsection{Encoding}

To encode the message, we use a specific message frame. This frame ensures that if part of the message is incorrectly transmitted, such as when a bit in the message is flipped, the receiver will detect the error and request the message to be resent until it is received correctly. This method is known as Automatic Repeat reQuest (ARQ).

The message includes two verification mechanisms, unit sum check and parity bit: one to check the integrity of the transmitted message and another to confirm the identity of the sender.

$$
\underbrace{\smallsmile \smallsmile \smallsmile \smallsmile}_{\text{ID of the sender}}~~
\underbrace{\smallsmile \smallsmile \smallsmile \smallsmile}_{\text{Number of Receivers}}~~
\underbrace{\smallsmile \smallsmile \smallsmile \smallsmile \dots \smallsmile \smallsmile \smallsmile \smallsmile}_{4 \times \text{Number of Receivers}}~~
\underbrace{\smallsmile \smallsmile \smallsmile \smallsmile \dots \smallsmile \smallsmile \smallsmile \smallsmile}_{\text{Message}}
~~~ \cdots
$$
$$
\cdots ~~~
\underbrace{\smallsmile \smallsmile \smallsmile \smallsmile}_{\text{Message end indicator}}\;
\underbrace{\smallsmile \smallsmile \smallsmile \smallsmile}_{\text{Units sum}}~~
\underbrace{\smallsmile \smallsmile \smallsmile \smallsmile}_{\text{Sender ID with parity bit}}
$$

\subsubsection{Description}

\textbf{Emitter ID:} $\smallsmile \smallsmile \smallsmile \smallsmile$\\
The unique 4-bit identifier for the sender of the message (range: 0-15).\\

\textbf{Number of Receivers:} $\smallsmile \smallsmile \smallsmile \smallsmile$\\
Optional 4-bit field indicating the number of receivers. If there is only one receiver, this field is set to \textit{0000}.\\

\textbf{Receiver IDs:} $\smallsmile \smallsmile \smallsmile \smallsmile \dots \smallsmile \smallsmile \smallsmile \smallsmile$\\
Each receiver's ID is encoded in 4 bits. For multiple receivers, these IDs are listed sequentially.\\

\textbf{Message Content:} $\smallsmile \smallsmile \smallsmile \smallsmile \dots \smallsmile \smallsmile \smallsmile \smallsmile$\\

Type indicators (up to 15 indicator types, \texttt{0000} being already reserved):\\
\begin{table}[H]
	\centering
	\begin{tabular}{c l l}
		\hline
		Type indicator & Type & Data size\\
		\hline
		\texttt{0000} & Message end & \texttt{4} bits\\
		\texttt{0001} & Integer & \texttt{8} bits\\
		\texttt{0010} & Integer & \texttt{32} bits\\
		\texttt{0011} & Float & \texttt{32} bits\\
		\texttt{0100} & String & \texttt{8} bits for length + \texttt{8} bits per character\\
		\texttt{0101} & List of int8 & \texttt{16} bits for length + \texttt{8} bits per integer\\
		\texttt{0110} & List of float & \texttt{16} bits for length + \texttt{32} bits per float\\
		\texttt{0111} & Message type ID & \texttt{8} bits\\
		\texttt{1000} & Robot info & \texttt{168} bits\\
		\texttt{1001} & & \\
		\texttt{1010} & Binary message & \texttt{16} bits for length + \texttt{1} bit per binary bit\\
		\texttt{1011} & & \\
		\texttt{1100} & & \\
		\texttt{1101} & & \\
		\texttt{1110} & & \\
		\texttt{1111} & & \\
		\hline
	\end{tabular}
	\caption{}
\end{table}

\textbf{Units sum:} $\smallsmile \smallsmile \smallsmile \smallsmile$\\
\texttt{4} bits representing the sum of the bit in the message frame modulo 16. \\ 

\textbf{Emitter ID with Parity:} $\smallsmile \smallsmile \smallsmile \smallsmile$\\
The emitter ID is repeated with a parity bit. The parity ensures the total number of 1s in the message frame up to the emitter ID with parity is correct. If odd, the last bit is flipped. \\



\subsection{Dencoding}


\subsection{Implementation}

To implement it, we define a table of message with their specification, for each message received, there is a frame of response with some test.


Message ID table:
\begin{table}[H]
	\centering
	\begin{tabular}{c l}
		\hline
		\textbf{Message ID} & \textbf{Description} \\
		\hline
		\texttt{00000000} & Ask to repeat the last message\\
		\texttt{00000001} & Last message received correctly\\[5pt]
		
		\texttt{00010000} & Ask the connected robot set\\
		\texttt{00010001} & Send the connected robot set\\[5pt]
		
		\texttt{00100000} & Ask to all where they want to go\\
		\texttt{00100001} & Send next waypoint position with frontier ID\\
		\texttt{00100010} & Send robot info\\
		\texttt{00100011} & Send next position to reach\\[5pt]
		
		\texttt{00110000} & Ask for live grid map synchronisation\\
		\texttt{00110001} & Send updated live grid map\\[5pt]
		
		\texttt{01110000} & Ask info gainned since last connection\\
		\texttt{01110001} & Send nothing gainned\\
		\texttt{01110010} & Send what was gainned on the live-grid map\\[5pt]

		\texttt{01010101} & Transmit message from another robot\\
		\hline
	\end{tabular}
\end{table}


Information sent for each message type could be found in the \textcolor{red}{\textbf{Appendix ...}}.

Robot info:
2 \textit{float32} for position, 1 \textit{float32} for velocity, 1 \textit{float32} for rotation, 1 \textit{float32} for angular velocity, 1 \textit{int8} for energy\\

Connected robot set:
\texttt{4} bits for set length + \texttt{4} bits per robot id



\subsection{Analysis scheme}

\subsubsection{Delimited range of application}
Can be for all type of environment

\subsection{Robustness to failure and uncertainty}
Must handle all possible cases, for instance if a robot is in a dead lock and doesn't succed in exit it

\subsection{Completeness}
Must explore all the domain or accomplish all the task given to it

\subsection{Effectiveness}
Must spread the work efficiently

\subsection{Speed}
Must be a rapid algorithm

\end{document}