\documentclass[aspectratio=169,10pt]{beamer}

% Import all packages
%%%%%%%%%%%%%%%%%%%%%%%%%%%%%%%
%     import des packages     %
%%%%%%%%%%%%%%%%%%%%%%%%%%%%%%%
\usepackage[export]{adjustbox}
\usepackage{amsmath,amsfonts,amssymb}
\usepackage{anyfontsize}
\usepackage{array}
\usepackage[french]{babel}
\usepackage{bbm}
\usepackage{colortbl}
\usepackage{comment}
\usepackage{cclicenses}
\usepackage{eqnarray}
\usepackage{eso-pic}
\usepackage{enumerate} % Pour personnaliser les énumération
\usepackage{fancybox}
\usepackage{fancyhdr}
\usepackage{float}
\usepackage[T1]{fontenc} 
\usepackage{forest}
\usepackage{gensymb}
\usepackage{geometry}
\usepackage{glossaries}
\usepackage{graphicx}
\usepackage{hyperref}
\usepackage{ifthen}
\usepackage{import}
\usepackage{indentfirst}
\usepackage[utf8]{inputenc}
\usepackage{lastpage}
\usepackage{libertine}
\usepackage{lipsum}
\usepackage{listings}
\usepackage{lmodern}
\usepackage{mathtools}
\usepackage{mdframed}
\usepackage{multicol}
\usepackage{pdfpages}
\usepackage{pifont}
\usepackage{stmaryrd}
\usepackage{subcaption}
\usepackage{subfiles}
\usepackage{tabularx}
% \usepackage{tcolorbox}
\usepackage[most]{tcolorbox}
\usepackage[absolute,overlay]{textpos} % To position the image
\usepackage{textcomp}
\usepackage{ulem}
\usepackage{wrapfig}
\usepackage{xcolor}


\newcommand{\titre}{Conception et optimisation d'algorithmes de coordination multi-robot}
\newcommand{\titrefooter}{Présentation - Projet de fin d'étude}
\newcommand{\soustitre}{Exploration autonome de réseaux de galerie}
\newcommand{\auteur}{Fabien MATHÉ}
\newcommand{\referent}{M. Mehmet ERSOY}
\newcommand{\institut}{Seatech - MOCA 3A}

% Thème clair et épuré
\usetheme{Berlin} % Thème Berlin
\usecolortheme{seahorse}
\setbeamertemplate{navigation symbols}{} % Suppression des symboles de navigation

% Personnalisation du thème
\include{personalisation}

% Titre de la présentation
\title[Titre court]{\titre}
\subtitle{\soustitre}
\author{\auteur}
\institute{\institut}
\date{\today}

\begin{document}

% Page de titre (Timeline non affichée ici)
\begin{frame}
    \titlepage
\end{frame}

% Table des matières
\begin{frame}{\textbf{Sommaire}}
    \tableofcontents
\end{frame}

% Section 1 : Introduction
\section{Présentation générale}
\begin{frame}{\textbf{Présentation générale}}
    \textbf{Simulateur robotique :}\\ 

    \begin{itemize}
        \item[] Code conçu pour reproduire de manière réaliste les comportements de robots dans un environnement virtuel.\\
        \item[] Utilisé pour concevoir, tester et valider des algorithmes de contrôle sans avoir à accéder à du matériel physique.\\
    \end{itemize}
    
\end{frame}

\begin{frame}{\textbf{Architecture générale}}
    \textbf{Simulateur simple 2D :\\}
    Utilisation de \textit{Python}, fenêtre graphique avec la librairie \textit{Pygame}.\\

    \textbf{Physique simplifiée :}\\
    

    Le robot est modélisé par un
    
\end{frame}


% Section 2 : Méthodologie
\section{Méthodologie}

\begin{frame}{\textbf{Méthodologie}}
    Un script principal : \textit{main.py}\\
    \hspace{1em}Initialise la simulation\\
    \hspace{2em}Initialise la carte\\
    \hspace{2em}Initialise le robot\\
    
    \hspace{1em}Ordonne les différentes actions du robot

\end{frame}

\begin{frame}{\textbf{Méthodologie}}
    Slide 4
\end{frame}

% Section 3 : Résultats
\section{Résultats}

\begin{frame}{\textbf{Résultats}}
    Slide 5
\end{frame}

\begin{frame}{\textbf{Résultats}}
    Slide 6
\end{frame}

\begin{frame}{\textbf{Résultats}}
    Slide 7
\end{frame}

\begin{frame}{\textbf{Résultats}}
    Slide 8
\end{frame}

\begin{frame}{\textbf{Résultats}}
    Slide 9
\end{frame}

% Section 4 : Discussions
\section{Discussions}

\begin{frame}{\textbf{Discussions}}
    Slide 10
\end{frame}

\begin{frame}{\textbf{Discussions}}
    Slide 11
\end{frame}


% Section 5 : Conclusion
\section{Conclusion}
\begin{frame}{\textbf{Conclusion}}
    Slide 12
\end{frame}


% Section 5 : Conclusion
\section{Références}
\begin{frame}{\textbf{Références}}
    Slide 13
\end{frame}

\end{document}
