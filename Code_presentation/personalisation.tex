
% Couleurs personnalisées
\definecolor{mainblue}{RGB}{0, 102, 204} % Bleu principal
\definecolor{lightgray}{RGB}{240, 240, 240} % Gris clair
\definecolor{darkgray}{RGB}{100, 100, 100} % Gris plus foncé pour les points non actifs
\definecolor{lightblue}{RGB}{173, 216, 230} % Bleu clair
\definecolor{darkblue}{RGB}{0, 51, 102} % Bleu foncé

% Configuration des couleurs
% \setbeamercolor{normal text}{bg=white, fg=black}
% \setbeamercolor{structure}{fg=mainblue}
% \setbeamercolor{frametitle}{bg=lightgray, fg=mainblue}
% \setbeamercolor{title}{fg=mainblue}
% \setbeamercolor{itemize item}{fg=mainblue}

% Personnalisation des couleurs du thème
\setbeamercolor{normal text}{bg=white, fg=black} % Texte principal en gris foncé sur fond blanc
\setbeamercolor{structure}{fg=mainblue} % Couleur principale des titres (sections, sous-sections)
\setbeamercolor{frametitle}{bg=lightgray, fg=mainblue} % Titre des frames avec fond bleu clair et texte bleu
\setbeamercolor{title}{fg=mainblue} % Couleur du titre principal
\setbeamercolor{itemize item}{fg=mainblue} % Couleur des puces (items)
\setbeamercolor{enumerate item}{fg=mainblue} % Couleur des items des listes numérotées
\setbeamercolor{section in toc}{fg=mainblue} % Sections dans la table des matières
\setbeamercolor{subsection in toc}{fg=darkblue} % Sous-sections dans la table des matières

% Personnalisation des blocs de couleur
\setbeamercolor{block title}{bg=mainblue, fg=white} % Titre des blocs avec fond bleu et texte blanc
\setbeamercolor{block body}{bg=lightblue, fg=black} % Corps des blocs avec fond bleu clair et texte noir
\setbeamercolor{alertblock title}{bg=darkblue, fg=white} % Titre des alertes avec fond bleu foncé et texte blanc
\setbeamercolor{alertblock body}{bg=lightblue, fg=black} % Corps des alertes avec fond bleu clair et texte noir
\setbeamercolor{exampleblock title}{bg=mainblue, fg=white} % Titre des exemples avec fond bleu et texte blanc
\setbeamercolor{exampleblock body}{bg=lightblue, fg=black} % Corps des exemples avec fond bleu clair et texte noir

% \setbeamercolor{palette primary}{use=structure,fg=white,bg=structure.fg}
% \setbeamercolor{palette secondary}{use=structure,fg=white,bg=structure.fg!75}
% \setbeamercolor{palette tertiary}{use=structure,fg=white,bg=mainblue}
% \setbeamercolor{palette quaternary}{fg=white,bg=mainblue}

% Modification des puces (bulles) dans les listes
\setbeamertemplate{itemize item}[circle] % Change les puces classiques en cercles
%\setbeamertemplate{itemize item}[square] % Change les puces en carrés
%\setbeamertemplate{itemize item}[triangle] % Change les puces en triangles
%\setbeamertemplate{itemize item}[rectangle] % Change les puces en rectangles

% Modification des numéros dans les listes numérotées (enumerate)
\setbeamertemplate{enumerate item}[circle] % Change les numéros en cercles
%\setbeamertemplate{enumerate item}[square] % Change les numéros en carrés
%\setbeamertemplate{enumerate item}[triangle] % Change les numéros en triangles
%\setbeamertemplate{enumerate item}[rectangle] % Change les numéros en rectangles


% Bas de page personnalisé (nom, établissement, date, numéro de page)
\setbeamertemplate{footline}{
    \begin{beamercolorbox}[ht=2ex, dp=1ex]{}
        \colorbox{lightgray}{
            \parbox{\textwidth}{
                \begin{minipage}{0.19\linewidth}
                    \centering
                    \auteur
                \end{minipage}
                \hfill
                \begin{minipage}{0.19\linewidth}
                    \centering
                    \institut
                \end{minipage}
                \hfill
                \begin{minipage}{0.19\linewidth}
                    \centering
                    \titrefooter
                \end{minipage}
                \hfill
                \begin{minipage}{0.19\linewidth}
                    \centering
                    \today
                \end{minipage}
                \hfill
                \begin{minipage}{0.19\linewidth}
                    \centering
                    \insertframenumber/\inserttotalframenumber
                \end{minipage}
                \vspace{0.5em}
            }
        }
    \end{beamercolorbox}
}
